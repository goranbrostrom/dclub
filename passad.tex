\chapter{I tredje och fjärde hand}

\section{Inledning}

När man överväger en tredje- eller fjärdehandsöppning måste man ta
hänsyn till att partnern {\em inte} öppnade. Han kan alltså inte ha
mer än 8-9 hp. Han har inte heller
en svag tvåöppning i högfärg. Därför föreskriver 
systemet att klöveröppningen är tre poäng starkare än  i första och andra
hand, 12--14 hp balanserad eller 18+ hp, alla fördelningar.. 

Styrkeintervallet för öppningarna \ru{1}, \ho{1} är alltså
ca 10--17 hp i tredje hand och ca 12--17 i fjärde hand.

Sangöppningen är på 15--17 hp och följer samma system som tidigare, även om
slaminviter inte är så frekventa. 

\subsection{Öppningen \kl{1}}

Visar alltså 12--14 hp och balanserad hand (femkorts högfärg är tillåtet)
eller 18+ hp, alla fördelningar. Svaren är så långt möjligt samma som efter
öppning i första och andra hand, eller

\bbe

\item[--\ru{1}] 0--5(6) hp, alla fördelningar. Sex bra poäng innehåller två
  kungar, ett ess och en dam, en kung och dam-knekt i samma färg, etc, dvs
  inte fyra knektar och en dam. Typ.

\item[--\hj{1}] 6--8 hp, (a) femkorts spader eller (b) balanserad eller
  semibalanserad, eventuellt med en dålig femkorts lågfärg eller \marmic.

\item[--\spa{1}] 6--8 hp, 5+ klöver.

\item[--NT{1}] 6--8 hp, femkorts hjärter.

\item[--\kl{2}] 6--8 hp, 5+ ruter.

\item[--\ru{2}] 6--8 hp, fyrkorts hjärter och 5+ (bra) lågfärg.
  
\item[--\hj{2}] 6--8 hp, fyrkorts spader och 5+ (bra) lågfärg.

\item[--\spa{2}] 6--8 hp, 5--5 i lågfärgerna.
    
\ebe

Fortsatt budgivning analogt med tidigare.
