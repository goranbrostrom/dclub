\chapter*{F\"orord}

%{\em D-club} är en utveckling av förlagorna
%\emph{Superklöver} och \emph{SuperFrid}.

Systemet  baseras på svagare
en\-öpp\-ning\-ar än i standardsystem. Den undre gränsen är {\em 8--9
hp}, och vi gör en motsvarade uppgradering med ca två poäng av
svarsbuden, {\em då anpassningen är oklar}. Det har visat sig fungera
alldeles utmärkt i praktiken. Då vi har anpassning bjuder vi snabbt till
den nivå, som föreskrivs av {\sc Lagen} (om totala antalet stick), medan
vi ligger lågt utan anpassning. I konkurrensbudgivning är metoderna särskilt
effektiva. Både ruteröppningen och högfärgsöppningarna lovar obalanserade
händer, ruteröppningen med minst fyrkortsruter (längre klöver kan finnas)
och högfärgsöppningarna med minst fem kort i öppningsfärgen.

En annan hörnpelare i systemet är den {\em dubbeltydiga klöveröppningen},
som visar antingen minst 15 hp oavsett fördelning, eller 9--11 hp,
balanserad hand, eventuellt med femkorts högfärg, eller \marmic\ med
singelruter. Detta öppningsbud är (förstås!) 
mycket frekvent, och vi har lagt ner stor möda på den fortsatta
budgivningen.

Sangöppningen är svag, 12--14 hp, och den kan innehålla en femkorts
högfärg. Därmed finns alla balanserade händer i 
öppningsbuden \kl{1}, \NT{1} och \NT{2}.
Genom att marmichänder med singelruter och 9---14 hp har speciella
öppningsbud (\kl{1} och \ru{2}), kan vi spela med {\em
femkorts högfärgsöppningar}.

Vad som hittills sagts gäller öppningar i första och andra hand. I {\em tredje
och fjärde hand} är betingelserna andra, genom att partnern har
begränsat sin styrka till max ca nio hp. Vi tar hänsyn till detta genom att
höja poänggränserna för en del specialbud.

\subsection{Nyheter}

Notera att vissa ``nyheter'' i tidiga versioner kan bli upphävda i senare.
\subsubsection*{Version 0.3}

Följande är nytt sen sommaren 2012:

\begin{itemize}
\item Oppningen \kl{1}:
\begin{itemize}
   \item \kl{1}--\hj{2}(\spa{2}) visar nu ca 9--13 hp och
     sexkorts (nån enstaka gång femkorts)  hjärter. Det är avlägg mot lilla handen (som dock
     får höja med bra 
     stöd). Krav till utgång mot starka handen.

     Konsekvenser:
\begin{itemize}
\item \kl{1}--\hj{1}; \NT{1}--\hj{2} är krav till utgång med 5+ hjärter
  (inviter via \kl{2}).
\item Efter \kl{1}--\kl{2}(\ru{2}) kan svararen ha fyrkorts högfärg
  \emph{med ut\-gångs\-styrka} mot lilla handen.
\end{itemize}
\end{itemize}
\item Enöppningar i tredje och fjärde hand:
\begin{itemize}
\item Vi höjer övre gränsen med 3
  hp, men den undre är densamma.
\item Svar högre än \NT{1} visar alltid trumfstöd. På tvåläget med
  sidofärg, högre (tretricksnivån) är splinter.
\end{itemize}
\item Tvåöppningarna är nya, se kommande sidor.
\item Sangbudgivningen är något modifierad, framför allt att man med 4--4(5) i
  hö och invitstyrka måste bjuda \ru{2}; \hj{2}--\spa{2}.
\item Hur man bjuder balanserade händer är modifierat (ingen naturlig
  \NT{2}-öppning). 
\end{itemize}

\subsection*{Version 0.4}

Från 1 januari 2013.

\begin{enumerate}
\item Svarsstrukturen efter klöveröppningen är reformerad. Följande gäller
  efter de olika svarsbuden.
\begin{description}
\item[\ru{1}:] Inget nytt.
\item[\hj{1}:] 9+ hp, 4+ hjärter, längre lågfärg och fyrkorts spader kan
  finnas. Efter öppnarens \NT{1} är \kl{2} försenad Stayman, \hj{2} avlägg. 
Öppnarens \spa{1} är entydigt 9--11 hp.
\item[\spa{1}:] Analogt med fortsättningen efter svaret \hj{1}.
\item[\NT{1}:] Inget nytt.
\item[\la{2}:] 9--14 hp, 5+ färg, förnekar fyrkorts högfärg. Ej krav.
\item[\hj{2}:] Minst 5--4 eller 4--5 i lågfärgerna, minst invit till utgång.
\item[\spa{2}:] Enfärgshand med en lågfärg eller balanserad utan högfärg,
  krav till utgång.
\end{description}
Notera att svarsbuden \hj{2} och \spa{2} inte finns efter pass i förhand.

\item Öppningsbudet \hj{2} tillåter fördelningen \ford{3}{4}{1}{5} med
  trekorts spader.

\item Fortsättningen efter \kl{1}--\ru{1}; \spa{1} är förenklad.

\item Fortsättningen efter \kl{1}--\hj{1}/spa{1}; \spa{1}/\NT{1} är förändrad.

\item \emph{Notera} fortsättningen efter \kl{1}--\ho{1}; \kl{2}

\item Svaren \kl{1}--\la{3} är numera en naturlig sanginvit mot
  \emph{svaga} handen, cirka 13--14 hp, bra sexkortsfärg (två topphonnörer).
\end{enumerate}

\subsection*{Version 0.8}

\begin{enumerate}

\item Ny struktur på spärröppningarna (\ru{2} och uppåt).

\item \kl{1}--\la{2} förtydligat, avlägg mot svaga handen, ca 10--13 hp.

\item \kl{1}--\hj{2} lovar minst 5--4/4--5 i lågfärgerna.

\item \kl{1}--\spa{2} (i) krav till utgång med lång lågfärg (enfärgshand)
  eller (ii) balanserad slaminvit (mot 9--11 hp) utan högfärger.

\item \kl{1}--\NT{3} återinfört (inget slamintresse mot 9--11 hp).

\item Tvåöppningarna är reviderade.

\item Kickback är slopat.
\end{enumerate}

\subsection*{Version 0.9}

Denna versionsuppdatering förbereder för version 1.0 genom att samla ihop
de lösa trådar som f.n.\ (29 augusti 2014) finns alldeles för många av.

\subsection*{Version 0.11}

Rättat till en del saker i konkurrensbudgivning, efter egna inkliv och
efter motståndarnas. 
