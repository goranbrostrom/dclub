\chapter{Kl\"over\"oppningen}

\"Oppningsbudet \kl{1} \"ar svagt eller starkt; i {\em f\"orsta och andra hand}
antingen 9--11 hp och balanserad hand eller 15+ hp med godtycklig f\"ordelning.

{\em I tredje och fj{\"a}rde hand}
 visar \kl{1} 12--14 hp och balanserad eller 18+ hp oavsett fördelning.
Se kapitlet om tredje- och fj\"ardehands\"oppningar f\"or detaljer. (Kan
också behöva uppdateras!)

\section{Svarsbud}

F{\"o}r opassad hand g{\"a}ller:

\bbe
   \item[--\ru{1}] 0--8(9) hp, alla fördelningar.

   \item[--\hj{1}] 9--14 hp (högst invit mot 9--1), alla fördelningar, se
     dock \ho{2}\la{2} nedan.

   \item[--\spa{1}] 15-17 hp, obalanserad, eller 18+, alla fördelningar.
   \item[--\NT{1}] 15--17 hp, balanserad.
   \item[--\la{2}\ho{2}] 9--13 hp, bra sexkortsfärg. Avlägg mot 9--11, men
     denne får höja med superanpassning.

   \item[--\NT{2}] 9--13 hp, minst 5--5 i lågfärgerna.

   \item[--\la{3}] Minst {\em bra} sexkortsf\"arg, två
     topphonnörer, invit mot 9--11.

   \item[--\ho{3}] 6--8 hp, minst bra sexkortsf\"arg, helst sju. {\em Stark}
                    utg{\aa}ngsinvit mot starka \"oppnaren.

\ebe

\subsection{St{\"o}rd kl{\"o}verbudgivning}

\emph{(Som förut)}

Den generella överföringsprincipen gäller efter dubbelt, \ru{1}, och
\hj{1}; överföringen visar vad budet 
över skulle ha betytt i ostörd budgivning!, Sangbud visar tvåfärgshänder
med de lägsta objudna färgerna. 

Svararen kan med \spa{1} (eller D på \spa{1}) tvinga öppnaren att bjuda
\NT{1} med svaga handen och starta en vanlig sangbudgivning.
Alltså,
efter direkta bud p{\aa} \kl{1}:

\begin{longtable}{l|lp{6cm}}

\sf Inkliv & Svarsbud & Betydelse \\ \hline

D      & \em RD         & 9+, \emph{klöver}. Öppnarens \pass\ visar svaga handen.\\

       & \ru{1}, \hj{1} & Visar f{\"a}rgen över, krav till utg{\aa}ng
      mot starka handen. Svaga handen f{\aa}r
         bara acceptera överföringen (2--3 kort) eller  h{\"o}ja enkelt
      (fyrkortsstöd).\\ 
      & \spa{1}         & 9+ hp, öppnaren måste bjuda \NT{1} med svaga handen,
      varefter normal sangbudgivning tar vid. \\
       & \NT{1} & 9+, 5--5 i lågfärgerna.\\
       & \kl{2} & 9+ hp, ruter.\\
       & \ru{2}, \hj{2} & 9--13 hp, minst sex kort i färgen över.\\
       & \spa{2} & Överföring till \emph{klöver}. \\
       & \NT{2}  & Svagt, minst 5--5 i l{\aa}gf{\"a}rgerna.\\ \hline
\ru{1}, \hj{1} & D & Överföring, minst fyra kort i färgen över. \\
            & \ho{1}, \NT{1} & Som efter dubbling.\\
           & \la{2}, \hj{2} & Överföring .\\
           & \spa{2} & Överföring till \emph{klöver}.\\ 
           & \NT{2} & Svag 5--5 i lägsta objudna. \\ \hline
\spa{1}  & D & 9+, bjud \NT{1} med svaga. \\ 
         & \NT{1} & 9+, 5--5 i lågfärgerna \\
         & \la{2}, \hj{2} & överföring \\ 
         & \spa{2} & Överföring till \emph{klöver}. \\
         & \NT{2} & Svag 5--5 i lågfärgerna. \\\hline
\end{longtable}

P{\aa} inkliv efter \kl{1}--\ru{1} bjuder vi enligt vanliga defensivprinciper, 
som om fienden har {\"o}ppnat. Utan överföring tills vidare!
{\"O}ppnaren m{\aa}sta alltid passa med den svaga 
varianten, men kan ocks{\aa} bli tvungen att passa med den 
starka och ol{\"a}mplig
f{\"o}rdelning.

P{\aa} inkliv efter \kl{1}--\ru{1}; \hj{1} visar bud 
ca 6--8, generella överföringsprinciper med dubbelt visande tangentfärgen..

\section{\kl{1} -- \ru{1}}

\emph{(Som förut)}

\subsection{\"Oppnarens {\aa}terbud}

\bbe
   \item[pass] 9--11 hp, tvunget.
   \item[\hj{1}] Konventionellt visande extrastyrka, ca 18+ hp.
   \item[\spa{1}] 4+ spader, obalanserad. Längre sidofärg kan alltså finnas.
   \item[\NT{1}] 15--17 hp, balanserad.
   \item[\la{2}, \hj{2}] Naturligt, passbart. Minst femkortsf\"arg,
   förnekar fyrkorts spader.
   %\item[\spa{2}] 15--17 mp, 5--5 i spader och ruter. 
   %\item[\NT{2}] 15--17 mp, 5--5 i klöver och en högfärg. \ru{3} frågar efter
%högfärgen.
   %\item[\la{3}] 15--17 mp, 5--5 i bjuden f\"arg och färgen över.
   %\item[\hj{3}] 15--17 mp, 5--5 i h\"ogf\"argerna.
\ebe

\subsection{\kl{1} -- \ru{1}; \hj{1}}

\bbe
   \item[\spa{1}] 0--6 hp, h\"ogst invit mot 18--20.
   \item[\NT{1}] 7--8 hp, 5+ \emph{spader}, krav till utg{\aa}ng.
   \item[\la{2} \hj{2}] 6--8 hp, minst fem kort i f\"argen. Krav 
                        till utg{\aa}ng.
   \item[\spa{2}] 7--8 hp, balanserad. Öppnaren g\"or sig till ``kapten''
                  med \NT{2} (t. ex. med 24+ utan femkortsf\"arger),
                  bjuder annars som svarare till en
                  \NT{2}-\"oppnare. Vanlig \NT{2}-budgivning i båda fallen.
   \item[\NT{2}] 7--8 hp, minst 5--5 i l{\aa}gf\"argerna.
   \item[\la{3}, \ho{3}] 6--8 hp, \ford{4}{4}{4}{1} med singel i
                  f\"argen \"over (\spa{3} med singelkl\"over). Endast Esset
                  ger po\"ang i singelf\"argen!
                  Krav till utg{\aa}ng. Öppnaren fr{\aa}gar efter kontroller
                  med bud i singelf\"argen; svaren startar med 0 (noll).
\ebe

\subsubsection{\kl{1} -- \ru{1}; \hj{1} -- \spa{1}}

Efter \spa{1} har svararen ca 0--6 hp.

\bbe
   \item[\NT{1}] 18--20 hp, balanserad hand. Sangsystemet tr\"ader i
                 kraft.
   \item[\kl{2}] \begin{itemize}
                   \item 14- mp, 5+ klöver.
                   \item 25+ hp, balanserad.
                  \end{itemize}

   \item[\ru{2},\ho{2}] 14- mp, 5+ färg, ej passbart. 

   \item[\NT{2}] 21--22 hp, balanserad.
   \item[\kl{3}] 20--23 hp, \marmic\ med singelruter. Kontrollsvar börjar på
   sex.
   \item[\ru{3}] 20--23 hp, \marmic\ med singelhjärter.
   \item[\hj{3}] 20--23 hp, \marmic\ med singelspader.	
   \item[\spa{3}] 20--23 hp, \marmic\ med singelklöver.	
   \item[\NT{3}] ``To play''.
\ebe

\paragraph{\kl{1} -- \ru{1}; \hj{1} -- \spa{1}; \kl{2}}

\bbe
   \item[--\ru{2}] 0--3 ``AKQ''-poäng.
     \bbe
	\item[\hj{2}] 25+ hp, balanserad, eller klöver och hjärter, krav
     för en rond.
	\item[\spa{2}] 5+ klöver och 4 spader, krav.
	\item[\NT{2}]  balanserad 23--24 hp.
     \ebe
   \item[-Övrigt] naturligt, 4+ ``AKQ''-poäng.
\ebe

\subsection{\kl{1} -- \ru{1}; \spa{1}}

\bbe
\item[--pass] Mycket svagt, ca 0--4 hp, inget stöd.
\item[--\NT{1}] 5--8 hp, 0--3 spader. Naturlig fortsättning, men öppnaren
  kan med bra spader ``hoppa över'' en lägre fyrkortsfärg. 
\item[--\la{2}, \hj{2}] 5+ färg 5--8 hp.
\item[--\spa{2}] 4+ spaderstöd, 0--4 hp.
\item[--\NT{2}] 4+ spaderstöd, 5--8 hp, någon singelton. \kl{3} frågar var.
\item[--\spa{3}] 4+ spaderstöd, 5--8 hp.
\ebe
\section{\kl{1} -- \hj{1}}

\emph{(Nytt)}

Visar alltså 9--14(15) hp, nästan alla fördelningar. Svar:

\bbe
\item[\spa{1}] 15+ hp, balanserad eller marmic, typ. Naturlig fortsättning.
\item[\NT{1}] 9-11 hp, balanserad. Vanlig sangbudgivning.
\item[\la{2}\ho{2}] 15+ hp, minst femkortsfärg, naturlig budgivning.
\ebe


\subsection{\kl{1} -- \spa{1}}

\emph{(Nytt)}

Krav till utgång, balanserad eller marmictyp. Fortsatt budgivning är i
princip naturlig, öppnaren bjuder 

\bbe
\item[\NT{1}] med 9--11 hp, normal sangbudgivning (men krav till utgång) följer, 
\item[\kl{2}] med 15+ hp, balanserad, bjud fyrkortsfärger.
\item[\ru{2}\ho{2}\kl{3}] med 15+ och minst femkortsfärg, naturlig
  fortsättning. 
\ebe

\section{\kl{1} -- \NT{1}}

\emph{(Nytt)}

Visar balanserad 15--17 hp, krav till utgång. Vanliga sangbudgivningen.

\section{\kl{1} -- \la{2}\ho{2}}

\emph{(Nytt)} 

Visar 9--13 hp, sexkortsfärg, passbart. Ny färg, \NT{2} är 15+, krav till
utgång, stöd 9--11 hp med bra anpassning och stöd.
 
\section{\kl{1} -- \NT{2}}

\emph{(Nytt förslag?)}

Kanske 18--20 hp, balanserad? Vanlig \NT{2}-budgivning.
