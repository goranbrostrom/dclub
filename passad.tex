\chapter{I tredje och fjärde hand}

\section{Inledning}

När man överväger en tredje- eller fjärdehandsöppning måste man ta
hänsyn till att partnern {\em inte} öppnade. Han kan alltså inte ha
mer än 8-9 hp. Han har inte heller
en svag tvåöppning i högfärg. Därför föreskriver 
systemet att klöveröppningen är entydigt stark, minst 16 hp oavsett
fördelning. \NT{2} visar fortfarande 21-22 hp, balanserad.

Styrkeintervallet för öppningarna \ru{1}, \ho{1} är alltså
ca 10-15 hp i tredje hand och ca 12-15 i fjärde hand. Lovar inte mer än
fyra kort i färgen. \NT{1} visar minst femkorts klöver utan fyrkorts högfärg.  

Den fortsatta budgivningen är också förenklad.

\bbe

\item[- \NT{1}] visar 6-8 hp och en (tämligen) balanserad
hand, två-över-en visar ungefär samma styrka, men en bra, minst
femkorts- men helst sexkortsfärg.

\item[- \NT{2}] visar en stark stödhand, femkortsfärg, även efter
ruteröppningen. Ca 7-8 hp, ``balanserad'' typ.

\item[- Dubbelhöjningen] baseras mer på fördelning och bra trumfstöd än
hon\-nörs\-styr\-ka, ca 0-5 hp, minst fyrkortsstöd. 

\item[- Hopp i ny] visar femkortsstöd och singel, bra kort.
\ebe

\section{Klöveröppningen}

Öppningsbudet \kl{1} är i tredje och fjärde hand entydigt starkt, minst 16
hp. Svararen bjuder nästan alltid \ru{1}, och den fortsatta budgivningen
blir precis som efter en öppning i första eller andra hand. 

(\emph Intressant fråga: vilka övriga svar kan tillåtas? Kanske pos med 8
fina poäng, krav till utgång?)

\section{Ruteröppningen}

Fortsatt budgivning är så mycket som möjligt som ''vanligt'':

\bbe
   \item[-\pass] 0-5 hp.
   \item[-\ho{1}] 6-8 hp, naturligt, längre klöver kan finnas.
      \bbe
        \item[\spa{1}] Naturligt.
        \item[\NT{1}] 5-4 eller 4-5 i lågfärgerna.
        \item[\kl{2}] Ruter är längsta färg.
        \item[\ru{2}] 11-15 hp, ruter och {\em hjärter}.
      \ebe
   \item[-\NT{1}] 6-8 hp, minst fyrkorts klöver, förnekar högfärger.
   \item[-\kl{2}] 6-10 hp, bra klöverfärg. Ingen fyrkorts högfärg.
   \item[-\ru{2}] En (svag) enkelhöjning utan fyrkorts högfärg.
   \item[-\ho{2}] Fyrkortsfärg och en {\em bra} enkelhöjning (minst 8 stp) med
                 minst fyr\-korts\-stöd.
   \item[-\NT{2}] Stark stödhand (1.5 hs?) utan fyrkorts högfärg.
   \item[-\kl{3}] Bra tvåfärgshand i lågfärgerna.
   \item[-\ho{3}, \kl{4}] Renons, femkortsstöd. Utan femkortsstöd kan man inte
                         ha till\-räck\-lig spelstyrka för så höga bud.
\ebe

\section{Högfärgsöppningarna}

\bbe
   \item[-\pass] 0-5 hp.
   \item[-\spa{1}] 6-8 hp, naturligt, längre lågfärg kan finnas.
   \item[-\NT{1}] 6-8 hp, tämligen balanserad.

   \item[-två-över-en] 6-8 hp, minst femkorts-, helst sexkortsfärg.
   \item[-enkel höjning] Svag stödhand.
   \item[-\NT{2}] Stenbergs, fyrkortsstöd, stark.
   \item[-\spa{2}, \la{3}] Hjärter(spader)-stöd och bra sidofärg eller
   singel\footnote{Vilket?}, bra kort. 
   \item[-dubbelhöjning] Svag spärr.
   \item[-Dubbelhopp i ny färg] Renons, minst femkortsstöd, maximum.
\ebe

\section{Sangöppningen}

Visar 10-15 hp, med klöver som enda färg eller med ruter som sidofärg. Obs,
kan vara en balanserad hand med klöver som enda färg.

\section{2 klöver}

Fortsatt budgivning:

\bbe
\item[-\ru{2}] Bjud högfärgen!
\item[-\ho{2}] Slutbudsförslag.
\item[-\NT{2} och högre] visar bra klöverstöd och ca 6-8 hp.
\ebe

