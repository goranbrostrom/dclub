\chapter{Kl\"over\"oppningen}

\"Oppningsbudet \kl{1} \"ar svagt eller starkt; i {\em f\"orsta och andra hand}
antingen 9--11 hp och balanserad hand eller 15+ hp med godtycklig f\"ordelning.

{\em I tredje och fj{\"a}rde hand}
 visar \kl{1} 12--14 hp och balanserad eller 18+ hp oavsett fördelning.
Se kapitlet om tredje- och fj\"ardehands\"oppningar f\"or detaljer.

\section{Svarsbud}

F{\"o}r opassad hand g{\"a}ller:

\bbe
   \item[--\ru{1}] 0--8(9) hp.

   \item[--\ho{1}] 9+ hp, minst fyrkortsf\"arg. L\"angre l{\aa}gf\"arg 
                  kan finnas.

   \item[--\NT{1}] 9--13(14) hp, balanserad utan fyrkorts h\"ogf\"arg.
               9--11--\"oppnaren ska passa.

   \item[--\la{2}] 9--13(14), minst sexkortsfärg utan högfärg, öppnaren får
     höja med 9--11 och bra stöd och bra kort, annars passbart. Se \la{3}
     för invithänder. 
        

   \item[--\hj{2}] 14+ hp, minst 4--4 i lågfärgerna utan högfärg. Minst invit
     mot svaga handen, om 4--4 så slamintresse.

   \item[--\spa{2}] Krav till utgång med balanserad utan högfärg eller
     enfärgshand med lågfärg.

   \item[--\NT{2}] 14--15 hp, balanserad utan h\"ogf\"arger. Invit.

   \item[--\la{3}] Minst {\em bra} sexkortsf\"arg, två
     topphonnörer,
                  invit mot 9--11.

   \item[--\ho{3}] 6--8 hp, minst bra sexkortsf\"arg, helst sju. {\em Stark}
                    utg{\aa}ngsinvit mot starka \"oppnaren.

   %\item[--\NT{3}] Finns inte längre.

   %\item[--\la{4}] Sydafrikansk Texas, dvs (n\"astan) g{\aa}ende f\"arg.
\ebe

\subsection{St{\"o}rd kl{\"o}verbudgivning}

Den generella överföringsprincipen gäller efter dubbelt, \ru{1}, och
\hj{1}; överföringen visar vad budet 
över skulle ha betytt i ostörd budgivning!, Sangbud visar tvåfärgshänder
med de lägsta objudna färgerna. 

Svararen kan med \spa{1} (eller D på \spa{1}) tvinga öppnaren att bjuda
\NT{1} med svaga handen och starta en vanlig sangbudgivning.
Alltså,
efter direkta bud p{\aa} \kl{1}:

\begin{longtable}{l|lp{6cm}}

\sf Inkliv & Svarsbud & Betydelse \\ \hline

D      & \em RD         & 9+, \emph{klöver}. Öppnarens \pass\ visar svaga handen.\\

       & \ru{1}, \hj{1} & Visar f{\"a}rgen över, krav till utg{\aa}ng
      mot starka handen. Svaga handen f{\aa}r
         bara acceptera överföringen (2--3 kort) eller  h{\"o}ja enkelt
      (fyrkortsstöd).\\ 
      & \spa{1}         & 9+ hp, öppnaren måste bjuda \NT{1} med svaga handen,
      varefter normal sangbudgivning tar vid. \\
       & \NT{1} & 9+, 5--5 i lågfärgerna.\\
       & \kl{2} & 9+ hp, ruter.\\
       & \ru{2}, \hj{2} & 9--13 hp, minst sex kort i färgen över.\\
       & \spa{2} & Överföring till \emph{klöver}. \\
       & \NT{2}  & Svagt, minst 5--5 i l{\aa}gf{\"a}rgerna.\\ \hline
\ru{1}, \hj{1} & D & Överföring, minst fyra kort i färgen över. \\
            & \ho{1}, \NT{1} & Som efter dubbling.\\
           & \la{2}, \hj{2} & Överföring .\\
           & \spa{2} & Överföring till \emph{klöver}.\\ 
           & \NT{2} & Svag 5--5 i lägsta objudna. \\ \hline
\spa{1}  & D & 9+, bjud \NT{1} med svaga. \\ 
         & \NT{1} & 9+, 5--5 i lågfärgerna \\
         & \la{2}, \hj{2} & överföring \\ 
         & \spa{2} & Överföring till \emph{klöver}. \\
         & \NT{2} & Svag 5--5 i lågfärgerna. \\\hline
\end{longtable}

P{\aa} inkliv efter \kl{1}--\ru{1} bjuder vi enligt vanliga defensivprinciper, 
som om fienden har {\"o}ppnat. Utan överföring tills vidare!
{\"O}ppnaren m{\aa}sta alltid passa med den svaga 
varianten, men kan ocks{\aa} bli tvungen att passa med den 
starka och ol{\"a}mplig
f{\"o}rdelning.

P{\aa} inkliv efter \kl{1}--\ru{1}; \hj{1} visar bud 
ca 6--8, generella överföringsprinciper med dubbelt visande tangentfärgen..

\section{\kl{1} -- \ru{1}}
\subsection{\"Oppnarens {\aa}terbud}

\bbe
   \item[pass] 9--11 hp, tvunget.
   \item[\hj{1}] Konventionellt visande extrastyrka, ca 18+ hp.
   \item[\spa{1}] 4+ spader, obalanserad. Längre sidofärg kan alltså finnas.
   \item[\NT{1}] 15--17 hp, balanserad.
   \item[\la{2}, \hj{2}] Naturligt, passbart. Minst femkortsf\"arg,
   förnekar fyrkorts spader.
   %\item[\spa{2}] 15--17 mp, 5--5 i spader och ruter. 
   %\item[\NT{2}] 15--17 mp, 5--5 i klöver och en högfärg. \ru{3} frågar efter
%högfärgen.
   %\item[\la{3}] 15--17 mp, 5--5 i bjuden f\"arg och färgen över.
   %\item[\hj{3}] 15--17 mp, 5--5 i h\"ogf\"argerna.
\ebe

\subsection{\kl{1} -- \ru{1}; \hj{1}}

\bbe
   \item[\spa{1}] 0--6 hp, h\"ogst invit mot 18--20.
   \item[\NT{1}] 7--8 hp, 5+ \emph{spader}, krav till utg{\aa}ng.
   \item[\la{2} \hj{2}] 6--8 hp, minst fem kort i f\"argen. Krav 
                        till utg{\aa}ng.
   \item[\spa{2}] 7--8 hp, balanserad. Öppnaren g\"or sig till ``kapten''
                  med \NT{2} (t. ex. med 24+ utan femkortsf\"arger),
                  bjuder annars som svarare till en
                  \NT{2}-\"oppnare. Vanlig \NT{2}-budgivning i båda fallen.
   \item[\NT{2}] 7--8 hp, minst 5--5 i l{\aa}gf\"argerna.
   \item[\la{3}, \ho{3}] 6--8 hp, \ford{4}{4}{4}{1} med singel i
                  f\"argen \"over (\spa{3} med singelkl\"over). Endast Esset
                  ger po\"ang i singelf\"argen!
                  Krav till utg{\aa}ng. Öppnaren fr{\aa}gar efter kontroller
                  med bud i singelf\"argen; svaren startar med 0 (noll).
\ebe

\subsubsection{\kl{1} -- \ru{1}; \hj{1} -- \spa{1}}

Efter \spa{1} har svararen ca 0--6 hp.

\bbe
   \item[\NT{1}] 18--20 hp, balanserad hand. Sangsystemet tr\"ader i
                 kraft.
   \item[\kl{2}] \begin{itemize}
                   \item 14- mp, 5+ klöver.
                   \item 25+ hp, balanserad.
                  \end{itemize}

   \item[\ru{2},\ho{2}] 14- mp, 5+ färg, ej passbart. 

   \item[\NT{2}] 21--22 hp, balanserad.
   \item[\kl{3}] 20--23 hp, \marmic\ med singelruter. Kontrollsvar börjar på
   sex.
   \item[\ru{3}] 20--23 hp, \marmic\ med singelhjärter.
   \item[\hj{3}] 20--23 hp, \marmic\ med singelspader.	
   \item[\spa{3}] 20--23 hp, \marmic\ med singelklöver.	
   \item[\NT{3}] ``To play''.
\ebe

\paragraph{\kl{1} -- \ru{1}; \hj{1} -- \spa{1}; \kl{2}}

\bbe
   \item[--\ru{2}] 0--3 ``AKQ''-poäng.
     \bbe
	\item[\hj{2}] 25+ hp, balanserad, eller klöver och hjärter, krav
     för en rond.
	\item[\spa{2}] 5+ klöver och 4 spader, krav.
	\item[\NT{2}]  balanserad 23--24 hp.
     \ebe
   \item[-Övrigt] naturligt, 4+ ``AKQ''-poäng.
\ebe

\subsection{\kl{1} -- \ru{1}; \spa{1}}

\bbe
\item[--pass] Mycket svagt, ca 0--4 hp, inget stöd.
\item[--\NT{1}] 5--8 hp, 0--3 spader. Naturlig fortsättning, men öppnaren
  kan med bra spader ``hoppa över'' en lägre fyrkortsfärg. 
\item[--\la{2}, \hj{2}] 5+ färg 5--8 hp.
\item[--\spa{2}] 4+ spaderstöd, 0--4 hp.
\item[--\NT{2}] 4+ spaderstöd, 5--8 hp, någon singelton. \kl{3} frågar var.
\item[--\spa{3}] 4+ spaderstöd, 5--8 hp.
\ebe
\section{\kl{1} -- \ho{1}}

Med 9--11 hp f{\aa}r \"oppnaren bjuda \spa{1} eller \NT{1}, eller g\"ora en
enkel h\"ojning. \kl{1}--\hj{1}; \spa{1} \"ar entydigt,
9--11 hp och 
fyr/femkorts\-f\"arg. I b\"agge fallen
f\"ornekas fyrkortsst\"od. \kl{1}-\hj{1}; \spa{2} lovar
15+ hp och minst femkortsf\"arg.

\subsection{\kl{1} -- \hj{1}; \spa{1}}

Visar alltså 9--11 hp och 4--5 spader, högst tre hjärter.

\bbe
   \item[--\NT{1}] Slutbud mot 9--11 hp. Naturlig fortsättning.
   \item[--\kl{2}] Försenad Stayman, minst utgångsinvit. Öppnaren redovisar
     ``extrakort'' i 
     högfärgerna nerifrån, \ru{2} utan extrakort (4--2 i hö).
   \item[--\ru{2}] Konventionellt krav till utgång med fyrkortshjärter. Kan
     innehålla fyrkorts spaderstöd. Öppnaren bjuder \hj{2} med min, \spa{2}
     med max, naturlig fortsättning. 
   \item[--\hj{2}] Avlägg.
   \item[--\spa{2}] Avlägg.
   \item[--\NT{2}] Naturlig invit.
   \item[--\la{3}] 4 hjärter och längre lågfärg, utgångsinvit.
   \item[--\ho{3}] Naturlig invit.
\ebe

\subsection{\kl{1} -- \hj{1}; \NT{1}}

Öppnaren har 9--11 hp utan fyrkorts högfärg.

\bbe
   \item[--\kl{2}] Försenad Stayman, minst femkorts hjärter, minst
     utgångsinvit. Öppnaren visar trekorts hjärter och min med \hj{2}, max
     med \spa{2}, utan trekorts hjärter bjuds \ru{2}. 
   \item[--\ru{2}] Konventionellt krav till utgång med fyrkortshjärter. 
        Öppnaren bjuder \hj{2} med min, \spa{2} med max, naturlig
        fortsättning. 
   \item[--\hj{2}] Avlägg.
   \item[--\spa{2}] Naturligt, krav för en rond.
   \item[--\NT{2}] Naturlig invit.
   \item[--\la{3}] 4 hjärter och längre lågfärg, utgångsinvit.
   \item[--\hj{3}] Naturlig invit.
\ebe

\subsection{\kl{1} -- \spa{1}; \NT{1}}

Öppnaren har 9--11 hp.

\bbe
   \item[--\la{2}] Försenad Stayman, minst femkorts spader, minst
     utgångsinvit. Öppnaren bjuder \hj{2} med fyrkorts, \spa{2} utan
     fyrkorts hjärter men med trekorts spader, \ru{2} utan något av detta. 
   \item[--\ru{2}] Konventionellt krav till utgång med fyrkortsspader. 
        Öppnaren bjuder \hj{2} med min, \spa{2} med max, naturlig
        fortsättning. 
   \item[--\hj{2}] Naturligt, ej invit.

   \item[--\spa{2}] 5+ spader, slutbud.
   \item[--\NT{2}] Naturlig invit.
   \item[--\la{3}] 4 spader och längre lågfärg, utgångsinvit.
   \item[--\hj{3}] 5--5, invit.
   \item[--\spa{3}] Naturlig invit.
\ebe

\subsection{\kl{1} -- \hj{1}; \kl{2}}

F\"ornekar fyrkortsst\"od, fr{\aa}gar efter f\"ordelning. \"Oppnaren
har ingen egen femkortsf\"arg, utom m\"ojligen kl\"over.
\bbe
   \item[--\ru{2}] Neutralt, ej sexkorts hj\"arter, ej femkorts sidof\"arg.
                Kan vara 4--4 i h\"ogf\"argerna, ej 4--5 (\spa{2}).
                Naturlig forts\"attning:
      \bbe
         \item[\hj{2}] trekorts\--st\"od och femkorts klöver.
         \item[ny f\"arg] visar kl\"over och bjuden (sido)f\"arg.
         \item[\NT{2}] \"ar vilande med j\"amn hand eller {\marmic}.
         \item[\kl{3}] antyder sexkortsf\"arg.
       \ebe
   \item[--\hj{2}] Sexkortsf\"arg, sidof\"arg kan finnas. Ny f\"arg visar nu
               kl\"over plus f\"argen, \NT{2} vilande.
   \item[--\spa{2}] Fyrkorts spader och minst femkorts hj\"arter.
   \item[--\NT{2}] Femkorts {\em ruter}\/.
   \item[--\kl{3}] 4--5 i hj\"arter och kl\"over.
   \item[--\ru{3}] 4--6 i hj\"arter och ruter.
   \item[--\hj{3}] Minst sexkorts mycket bra hj\"arterf\"arg, enf\"argshand.
   \item[--\spa{3}] 5--6 i h\"ogf{\"a}rgerna, bra hand.
\ebe

Till{\"a}ggas b{\"o}r att budgivningen blir
helt analog om svararen bjuder \spa{1}
ist{\"a}llet f{\"o}r \hj{1}.

% \subsection{\kl{1} -- \ho{1}; \ru{2}}

% Lovar minst fyrkortsstöd, 9--11 eller 15+ hp. Svararen förutsätter
% 9--11, förstås.

% Fortsatt budgivning:
% \bbe
%    \item[--\ho{2}] Avlägg mot 9-11 hp.
%       \bbe
% 	\item[\pass] 9-11 hp.
% 	\item[\spa{2}, 3ny] 15+, stark femkortsfärg. 
% 	\item[\NT{2}] 15+, ej helt minimal. Svararen visar gedigen sidofärg
%       på tretricksnivån, kortfärg med hopp.
%         \item[\ho{3}] 15-16 hp, risig slamhand.
%         \item[hopp i ny] Kortfärg, tillägg. 
%        \ebe
%    \item[2 ahö] Krav till utgång mot 9-11 hp.
%    \item[\NT{2}] Invitstyrka (exakt!) mot 9-11, tämligen balanserad. Bud,
%       som tvingar till utgång är nu slaminvit, 15+ (utom utgångsbudet
%       självt!). 
%    \item[\la{3}] Positiv invit. Bud,
%       som tvingar till utgång är nu slaminvit, 15+ (utom utgångsbudet
%       självt!). 
%    \item[hoppi ny] Kortfärg, slaminvit mot 9-11 hp.
% \ebe
 	 
\subsection{Övriga starka bud efter \kl{1} - \ho{1}.}

Ny f\"arg \"ar natur\-ligt med 15+ hp, f\"or\-nekar fyrkorts trumf\-st{\"o}d.

\NT{2} lovar 15+ hp, 4+ stöd. Stenbergsmodul gäller, men krav till utgång
råder. 

\section{\kl{1} -- \NT{1}}

Naturlig forts\"attning; \pass\ med 9--11,
annars fyrkorts l{\aa}gf\"arger och femkorts h\"ogf\"arger.
\NT{2} ger svararen chansen att
visa upp en femkorts l{\aa}gf\"arg, visst slamintresse m{\aa}ste finnas.
Utan femkorts l{\aa}gf\"arg tolkar svararen \NT{2} som en kvantitativ invit
till slam i sang och bjuder d\"arefter (t.ex. \NT{3} med 9-10, annars
\ho{3} med gubbar, \NT{4} med supermax; \"oppnaren kan fortfarande
plocka fram en fyrkorts l{\aa}gf\"arg).

\section{\kl{1} -- \la{2}}

Visar 9+ hp, 5+ klöver. Förnekar fyrkorts högfärg. Öh passar eller höjer
med min. Övrigt naturligt, krav till utgång.

\section{\kl{1} -- \hj{2}}

Minst invit med minst 5--4/4--5 i lågfärgerna.
\bbe
\item[\spa{2}] Relä med svag (9--11 hp) eller tämligen balanserad stark
  (15+ hp). Svararen bjuder \la{3} med min, allt annat är utgångskrav:
  \NT{2} utan singel,  \ho{3} med singel.
\item[\NT{2}] 9--11 hp, håll i högfärgerna, ej krav.
\item[\la{3}] 9--11 hp, naturligt okrav.
\item[\ho{3}] 15+ hp, naturligt krav.
\item[\NT{3}] 9--11 hp, okrav.
\ebe

\section{\kl{1} -- \spa{2}}
  
Krav till utgång utan högfärg, begär att öppnaren bjuder \NT{2} för att få
reda på handtypen. Naturliga svar, \ho{3} visar singel och sexkorts
lågfärg, \NT{3} visar den balanserade handen och är \emph{krav}. 
\bbe
\item[\NT{2}] Begär förtydligande (med 9--11 eller 15+ hp):
\bbe
\item[-\la{3}] Sexkortsfärg. 
\item[-\hj{3}] Balanserad 24+ hp, naturlig fortsättning.
\item[-\spa{3}] Balanserad 22-23 hp, naturlig fortsättning.
\item[-\NT{3}] Balanserad 20-21 hp, \emph{okrav}. Naturlig fortsättning.
\ebe
\item[\la{3}, \ho{3}] Självgående trumffärg, 15+ hp.
\ebe 
 
\section{\kl{1} -- \NT{2}}
Med 9-11 hp bjuder \"oppnaren {\em pass} eller \NT{3}. Alla andra bud \"ar
minst 15 hp och naturligt; fyrkorts l{\aa}gf{\"a}rger
och femkorts h{\"o}gf{\"a}rger.
