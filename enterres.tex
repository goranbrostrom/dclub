
\documentstyle[a4wide]{article}
\begin{document}
\begin{verbatim}
{$N-,E-,S+,R+,I+,V+,F+,D+,L+}

UNIT EnterRes;

INTERFACE

USES CRT, CaGlobal, Util, EditLn;

Procedure EnterResults(VAR Info : ParData);

IMPLEMENTATION

PROCEDURE EnterResults(VAR Info : ParData);

VAR
   Nord, Ost         : ParNrSet;
   Poang             : INTEGER;
   RondNo,
   Start, Slut       : GivSet;
   tempAntRes        : ParNrSet;
   Active            : SET of ParNrSet;


PROCEDURE Nulla(VAR Bricka : Giv; Nord, Ost : ParNrSet);

BEGIN
   WITH Bricka DO
   BEGIN
      Point[Nord] := 0;
      Point[Ost] := 0;
      Mot[Nord] := 0;
      Mot[Ost] := 0;
      Grupp[Nord] := 0;
      Grupp[Ost] := 0;
      NS[Nord] := FALSE;
      NS[Ost] := FALSE;
      Score[Nord] := 0;
      Score[Ost] := 0;
      IMP[Nord] := 0;
      IMP[Ost] := 0;
      ScoreCorr[Nord] := 0;
      IMPCorr[Nord] := 0;
      ScoreCorr[Ost] := 0;
      IMPCorr[Ost] := 0;
      IF Nord IN Active THEN
         Active := Active - [Nord];
      IF Ost IN Active THEN
         Active := Active - [Ost];
   END;  { WITH }
END;  { Nulla }


PROCEDURE SkrivBrickor(Start, Slut : GivSet);

VAR
   j, AntRader          : BYTE;
   peka                 : ARRAY[ParNrSet] of ParNrSet;
   mm, RealAntPar,
   m, ParSlut, ParStart : ParNrSet;
   i                    : GivSet;
   ch                   : CHAR;
   Bricka               : Giv;

CONST
   AntParOnRad = 9;

BEGIN
   WITH Info DO
   BEGIN
      RealAntPar := 0;
      FOR m := 1 TO AntPar DO
         IF Deltar[m] THEN
         BEGIN
            INC(RealAntPar);
            peka[RealAntPar] := m;
         END;
      AntRader := RealAntPar DIV AntParOnRad;
      IF ((RealAntPar MOD AntParOnRad) <> 0) THEN INC(AntRader);
      FOR j := 1 TO AntRader DO
      BEGIN
         ParSlut := j * AntParOnRad;
         ParStart := ParSlut - AntParOnRad + 1;
         IF (j = AntRader) THEN ParSlut := RealAntPar;
         GoToXY(1, PRED(j) * (AntBrRo + 2) + 1);
         PutColor(MenuHiColor);
         WRITE('Bricka ');
         OpenDataFil;
         IF (FileSize(DataFil) < Start) THEN
         BEGIN
            FillChar(Bricka, SizeOf(Bricka), 0);
            for m := start to slut do
            begin
               seek(DataFil, (m-1));
               write(DataFil, Bricka);
            end;
         END
         ELSE
         BEGIN
            SEEK(DataFil, (Start - 1));
            READ(DataFil, Bricka);
         END;
         PutColor(CurrWindow.Win.Color);
         WITH Bricka DO
            FOR m := ParStart TO ParSlut DO
            BEGIN
               mm := peka[m];
               IF Mot[mm] = 0 THEN
                  WRITE('   ', mm:2, '  ')
               ELSE
               BEGIN
                  IF NS[mm] THEN PutColor(MenuRevColor);
                  WRITE(' ', mm:2, '-', Mot[mm]:2, ' ');
                  PutColor(CurrWindow.Win.Color);
               END;
            END;
         PutColor(CurrWindow.Win.Color);
         WRITELN;
         FOR i := Start TO Slut DO
         BEGIN
            SEEK(DataFil, (i-1));
            READ(DataFil, Bricka);
            PutColor(MenuHiColor);
            WRITE(i:3, ':   ');
            PutColor(CurrWindow.Win.Color);
            FOR m := ParStart TO ParSlut DO
            BEGIN
               mm := peka[m];
               IF (Bricka.Mot[mm] = 0) THEN
                  WRITE('  ---- ')
               ELSE
               BEGIN
                  IF (Bricka.Grupp[mm] > 1) THEN
                     PutColor(MenuRevColor);
                  WRITE((Bricka.Point[mm]):6, ' ');
                  PutColor(CurrWindow.Win.Color);
               END;
            END;
            WRITELN;
         END;
         CLOSE(DataFil);
      END;
   END;  { WITH }
   WRITELN;
END; { SkrivBrickor }

PROCEDURE UpDate(Start, Slut : GivSet);

PROCEDURE PrintHelpKeys;

VAR
   oldX, oldY : BYTE;

BEGIN
   oldX := WhereX;
   oldY := WhereY;
   GoToXY(15, CurrWindow.Win.height);
   PutColor(MenuRevColor);
   WRITE('F1');
   PutColor(CurrWindow.Win.Color);
   WRITE(' - terg     ');
   PutColor(MenuRevColor);
   WRITE('ESC');
   PutColor(CurrWindow.Win.Color);
   WRITE(' - Radera     ');
   PutColor(MenuRevColor);
   WRITE('Enter');
   PutColor(CurrWindow.Win.Color);
   WRITE(' - Acceptera');
   GoToXY(oldX, oldY);
END;  { PrintHelpKeys }

PROCEDURE GetParNr(VAR k : ParNrSet; VAR Result : CHAR);

VAR
   OK         : BOOLEAN;
   InString   : Str80;
   InReal     : REAL;
   felpos     : INTEGER;
   oldX,
   oldY, P    : BYTE;

BEGIN
   PutColor(MenuRevColor);
   oldX := WhereX;
   oldY := WhereY;
   WITH Info DO
   REPEAT
      InString := '';
      P := 0;
      EditLine(InString, 3, oldX, oldY, P, ['0'..'9'],
               [CR, ESC, F1], Result);
      CASE Result OF
         CR   : IF (InString = '') THEN
                   OK := FALSE
                ELSE
                BEGIN
                   VAL(InString, k, felpos);
                   IF Deltar[k] THEN
                   BEGIN
                      IF k IN Active THEN
                      BEGIN
                         ChangeCursor(OFF);
                         IF Prompter(NumText(k) + ' frekommer tv gnger.'
                                      + ' Fortstta nd') THEN
                            OK := TRUE
                         ELSE
                            OK := FALSE;
                         ChangeCursor(ON);
                         PrintHelpKeys;
                         PutColor(MenuRevColor);
                      END
                      ELSE
                      BEGIN
                         Active := Active + [k];
                         OK := TRUE;
                      END;
                   END
                   ELSE
                   BEGIN
                      ChangeCursor(OFF);
                      Message('Par nummer ' + NumText(k) +
                              ' deltar ej i denna tvling');
                      ChangeCursor(ON);
                      PrintHelpKeys;
                      PutColor(MenuRevColor);
                      OK := FALSE;
                   END
                END;
         ESC,
         F1   : OK := TRUE;
      END;  { CASE }
   UNTIL OK;
   PutColor(CurrWindow.Win.Color);
END;  { GetParNr }

PROCEDURE GivePoints(VAR Bricka : Giv; VAR Result : CHAR;
                     VAR AllGrupp : BYTE);

VAR
   InString   : Str80;
   InReal     : REAL;
   oldX, P,
   oldY       : BYTE;
   felpos     : INTEGER;

BEGIN
   oldX := WhereX;
   oldY := WhereY;
   felpos := 0;
   REPEAT
      PutColor(MenuRevColor);
      InString := '';
      P := 0;
      EditLine(InString, 5, oldX, oldY, P, ['-', '0'..'9'],
               [ESC, UpKey, DownKey, CR, F1], Result);
      PutColor(CurrWindow.Win.Color);
      IF Result IN [CR, F1] THEN
      BEGIN
         VAL(InString, Poang, felpos);
         IF (Poang MOD 10) <> 0 THEN
         BEGIN
            Beep;
            Result := NULL;
         END;
      END;
   UNTIL (Result IN [ESC, UpKey, DownKey, CR, F1]) AND (felpos = 0);
   WITH Bricka DO
   BEGIN
      CASE Result OF
       CR, F1    : BEGIN
                      Point[Nord] := Poang;
                      Point[Ost] := -Poang;
                      Grupp[Nord] := AllGrupp;
                      Grupp[Ost] := AllGrupp;
                      Mot[Nord] := Ost;
                      Mot[Ost] := Nord;
                      NS[Nord] := TRUE;
                      NS[Ost] := FALSE;
                   END;
         ESC     : BEGIN
                      Nulla(Bricka, Nord, Ost);
                   END;
      END;  { CASE }
   END;  { WITH }
END;  { GivePoints }


PROCEDURE OutGruppText(VAR AllGrupp : BYTE);

VAR
   oldX, oldY : BYTE;

BEGIN
   oldX := WhereX;
   oldY := WhereY;
   GoToXY(5, CurrWindow.Win.Height - 1);
   WRITE('Hgre Grupp ');
   PutColor(MenuRevColor);
   WRITE(#24);
   PutColor(CurrWindow.Win.Color);
   WRITE('  Lgre grupp ');
   PutColor(MenuRevColor);
   WRITE(#25);
   PutColor(CurrWindow.Win.Color);
   WRITE('    Nuvarande grupp = ', Allgrupp:3);
   GoToXY(oldX, OldY);
END;  {OutGruppText }

PROCEDURE AwayGruppText;

VAR
   oldX, oldY : BYTE;

BEGIN
   oldX := WhereX;
   oldY := WhereY;
   GoToXY(5, CurrWindow.Win.Height - 1);
   ClrEOL;
   GoToXY(oldX, OldY);
END;  { AwayGruppText }

PROCEDURE GetPoints(VAR AllGrupp : BYTE; VAR Finish : BOOLEAN);

VAR
   i      : GivSet;
   Result : CHAR;
   oldX,
   oldY   : BYTE;
   Bricka : Giv;

BEGIN
   OpenDataFil;
   SEEK(DataFil, (Start - 1));
   READ(DataFil, Bricka);
   IF Bricka.Mot[Ost] <> 0 THEN
   IF Bricka.Mot[Ost] <> Nord THEN
   BEGIN
      FOR i := Start TO Slut DO
      BEGIN
         SEEK(DataFil, (i-1));
         READ(DataFil, Bricka);
         Nulla(Bricka, Ost, Bricka.Mot[Ost]);
         SEEK(DataFil, (i-1));
         WRITE(DataFil, Bricka);
      END;
      SkrivBrickor(Start, Slut);
      Active := Active + [Ost];
   END;
   WRITELN;
   OpenDataFil;
   FOR i := Start TO SLUT DO
   BEGIN
      SEEK(DataFil, (i-1));
      READ(DataFil, Bricka);
      WITH Bricka DO
      BEGIN
         WRITELN;
         IF (MaxGrupp < 1) THEN
            MaxGrupp := 1;
         AllGrupp := 1;
         oldX := WhereX;
         oldY := WhereY;
         REPEAT
            GoToXY(oldX, oldY);
            WRITE('Bricka ', i:2, ' pong: ');
            OutGruppText(AllGrupp);
            GivePoints(Bricka, Result, AllGrupp);
            IF ((Result = UpKey) AND (AllGrupp < 100)) THEN
            BEGIN
               INC(AllGrupp);
               IF (MaxGrupp < AllGrupp) THEN
                  MaxGrupp := AllGrupp;
            END
            ELSE
               IF ((Result = DownKey) AND (AllGrupp > 1))THEN
                  DEC(AllGrupp);
         UNTIL (Result IN [ESC, CR, F1]);
      END;
      SEEK(DataFil, (i-1));
      WRITE(DataFil, Bricka);
   END;
   CLOSE(DataFil);
   ClrScr;
   SkrivBrickor(Start, Slut);
   IF (Result = F1) THEN
      Finish := TRUE
   ELSE
      PrintHelpKeys;
END;  { GetPoints }


VAR
   j          : ParNrSet;
   Result     : CHAR;
   nyX, nyY,
   oldX, oldY : BYTE;
   AllGrupp   : BYTE;
   i          : GivSet;
   OK, Finish : BOOLEAN;
   Bricka     : Giv;

BEGIN  { UpDate }
   Active := [];
   SkrivBrickor(Start, Slut);
   oldX := WhereX;
   oldY := WhereY;
   PrintHelpKeys;
   AllGrupp := 1;
   REPEAT
      GoToXY(oldX, oldY);
      ClrEOL;
      Finish := FALSE;
      WRITE(' NS par Nr: ');
      GetParNr(Nord, Result);
      Finish := FALSE;
      CASE Result of
         CR      : BEGIN
                      OpenDataFil;
                      IF (FileSize(DataFil) < Start) THEN
                      BEGIN
                      END;
                      SEEK(DataFil, (Start-1));
                      READ(DataFil, Bricka);
                      IF Bricka.Mot[Nord] <> 0 THEN
                      BEGIN
                         FOR i := Start TO Slut DO
                         BEGIN
                            SEEK(DataFil, (i-1));
                            READ(DAtaFil, Bricka);
                            Nulla(Bricka, Nord, Bricka.Mot[Nord]);
                            SEEK(DataFil, (i-1));
                            WRITE(DataFil, Bricka);
                         END;
                         nyX := WhereX;
                         nyY := WhereY;
                         SkrivBrickor(Start, Slut);
                         GoToXY(nyX, nyY);
                         Active := Active + [Nord];

                      END;
                      CLOSE(DataFil);
                      WRITE(', V par Nr: ');
                      GetParNr(Ost, Result);
                      CASE Result OF
                         CR  : IF (Nord <> Ost) THEN
                                  GetPoints(AllGrupp, Finish)
                               ELSE
                                  Active := Active - [Ost];
                         ESC : IF (Nord IN Active) THEN
                                  Active := Active - [Nord];
                         F1  : Finish := TRUE;
                      END;  { Case OstResult }
                   END;
         ESC     : BEGIN
                      ChangeCursor(OFF);
                      IF Prompter('Vill du verkligen stryka hela rond ' +
                                  NumText(RondNo)) THEN
                      BEGIN
                         Finish := TRUE;
                         OpenDataFil;
                         FOR i := Start TO Slut DO
                         BEGIN
                            SEEK(DataFil, (i-1));
                            FillChar(Bricka, SizeOf(Bricka), 0);
                            WRITE(DataFil, Bricka);
                         END;
                         CLOSE(DataFil);
                      END
                      ELSE
                      BEGIN
                         PrintHelpKeys;
                         ChangeCursor(ON);
                      END;
                   END;
         F1      : Finish := TRUE;
      END;  { Case NordResult }
   UNTIL Finish;
   FOR i := Start TO Slut DO
   BEGIN
      OpenDataFil;
      SEEK(DataFil, (i-1));
      READ(DataFil, Bricka);
      WITH Bricka DO
      BEGIN
         MaxGrupp := 1;
         FOR j := 1 TO Info.AntPar DO
            IF (MaxGrupp < Grupp[j]) THEN MaxGrupp := Grupp[j];
      END;
      CLOSE(DataFil);
      UpDatePoints(Bricka, i, Info);
   END;
END;  { UpDate }

VAR
   InString   : Str80;
   InReal     : REAL;
   Result     : CHAR;
   oldX, oldY : BYTE;

BEGIN  { EnterResults }
   IF (Info.AntRond = 0) THEN
   BEGIN
      Message('Ls in parnamnen frst');
      EXIT;
   END;
   FullScreenWindow.MakeWin;
   ChangeCursor(ON);
   ClrScr;
   Nord := 0;
   Ost := 0;
   Poang := 0;
   oldX := WhereX;
   oldY := WhereY;
   WITH Info DO
   BEGIN
      REPEAT
         GoToXY(oldX, oldY);
         WRITE(' ROND Nummer: ');
         InKey(WhereX, WhereY, 2, 'B', 'N', InString, InReal, Result);
         RondNo := TRUNC(InReal);
         IF NOT RondNo IN [1..AntRond] THEN Beep;
      UNTIL (Result = NULL) AND (RondNo IN [1..AntRond]);
      Slut := AntBrRo * RondNo;
      Start := Slut - AntBrRo + 1;
   END;  { WITH Info }
   UpDate(Start, Slut);
   UpDateFiles(Info);
   FullScreenWindow.MakeWin;
   HalfScreenWindow.MakeWin;
   ChangeCursor(OFF);
END;  { EnterResults }

BEGIN

END.  { UNIT EnterRes }
\end{verbatim}
\end{document}kloker[14] ~/bridge % 