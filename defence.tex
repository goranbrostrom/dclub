\chapter{Försvarsbudgivning}

Vi följer i allt väsentligt {\em Modern Standard}. Avvikelser presenteras
här.

\section{Allmänna principer}

\bbe
\item[``Scrambling'' \NT{2}]
\item[Bra \la{3} och dåliga \NT{2}]
\ebe
 
\section{Inkliv}


\vspace{.5cm}

\bbe

\item[Enkla inkliv] är som vanligt, men vi tar häsyn till att en passad partner
inte kan vara så stark som i {\em MS}.

\item[Inkliv med enkelt hopp] är starka, ca 12-16 hp och en bra
sexkortsfärg på två\-tricks\-nivån, något starkare på tretricksnivån.

\item[Inkliv med dubbelt hopp eller värre] är spärr.

\item[Ovanliga sanginkliv] I fjärde hand visar \NT{1} de båda objudna
färgerna, om motståndarna har bjudit en-över-en. Skälet att inte dubbla
eller bjuda \NT{2} är att man inte vill riskera att hamna på
tretricksnivån.
\ebe

Händer från ca 17 hp dubblar vi med, även om inte fördelningen är
den rätta.

\subsection{Fortsatt budgivning}

\subsubsection{Tredje hand passar eller dubblar negativt}

Vi spelar med Jeff Rubens \emph{useful space principle}, dvs färgbuden från
     och med överbudet tom \under{3} är \emph{överföringar}.
     \bbe
       \item[Ny färg] på tvåtricksnivån, under öppningsbudet, är
     konstruktivt,  men inte krav.
       \item[Dubbelhöjning] är \emph{inviterande}!
       \item[3 under] antingen starkt eller svagt, fyrkortsstöd.
 \item[Från och med enkelt överbud] är färgbud överföring till närmast högre
     färg. 

   \item[Sangbud] är naturliga, på tvåtricksnivån starkt inviterande.
     \ebe

\subsubsection{Tredje hand bjuder ny färg.}

Inga överföringar längre. \emph{Dubbelt} är negativt.

\section{Sangförsvar}

{\em Malmö} är det som gäller.

Vi balanserar {\em mycket frekvent} efter \NT{1}-\pass-\pass!

\section{Försvar mot svaga två\-öpp\-ning\-ar}

Modellen är {\em BTC}, dvs {\em Dubbelt} av \hj{2} lovar fyrkorts spader,
medan \NT{2} är en {\em UD} utan fyrkorts spader. Efter spärren \spa{2}
blir det tvärtom, fast frågan gäller hjärterinnehav eller ej.

Lebensohl används så att direkt färgbud är svagt, medan \NT{2} är
kommando till \kl{3} och lovar några poäng (minst ca 7 hp). Undantag:
\kl{3} lovar minst ca 7 hp.

\section{Försvar mot svaga tre\-öpp\-ning\-ar}

Normala \emph{UD}. \NT{3} är ``to play''.

\section{Försvar mot {\em Multi} \ru{2}}

{\em Dubbelt} lovar åtminstone en fyrkorts högfärg och en {\em
UD}-lämplig hand. Om tredje hand bjuder en högfärg så är fjärde hands
dubbling en UD, medan \pass\ är krav.

\section{Försvar mot stark klöver}

Huvudinriktningen  är destruktiv. Man bjuder dock även med bra
två\-färgs\-hän\-der:

\bbe
   \item[Dubbelt] visar klöver.
   \item[\NT{1}] en lågfärg och en högfärg.
   \item[\kl{2}] bägge högfärgerna.
   \item[\NT{2}] Bägge lågfärgerna.
\ebe

\section{Försvar mot dubbeltydig klöver.}

Ungefär som mot stark klöver. Med bra kort sätter man sig lämpligen i
busken. Färgbud på enläget dock konstruktiva.

