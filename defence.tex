\chapter{Försvarsbudgivning}

Vi följer i allt väsentligt {\em Modern Standard}. Avvikelser presenteras
här.

\section{Allmänna principer}

\bbe
\item[``Scrambling'' \NT{2}]
\item[Bra \la{3} och dåliga \NT{2}]
\ebe
 
\section{Inkliv}


\vspace{.5cm}

\bbe

\item[Enkla inkliv] är som vanligt (men kan vara starka, upp mot 18 hp),
  men vi tar häsyn till att en passad partner 
inte kan vara så stark som i {\em MS}.

\item[Inkliv med enkelt hopp] är spärr, 

\item[Inkliv med dubbelt hopp eller värre] är spärr.

\item[Ovanliga sanginkliv] I fjärde hand visar \NT{1} de båda objudna
färgerna, om motståndarna har bjudit en-över-en. Skälet att inte bjuda
\NT{2} är att man inte vill riskera att hamna på 
tretricksnivån.
\ebe

Händer från ca 19 hp dubblar vi med, även om inte fördelningen är
den rätta.

\subsection{Fortsatt budgivning efter enkelt inkliv}

\subsubsection{Tredje hand passar eller dubblar negativt}

Vi spelar med Jeff Rubens \emph{useful space principle}, dvs färgbuden från
     och med \emph{överbudet} till och med budet under \emph{inklivsfärgen}
     är \emph{överföringar}. 
     \bbe
       \item[Ny färg] under öppningsbudet är
     konstruktivt och krav.
       \item[Dubbelhöjning] är \emph{inviterande}!
       \item[3 under] antingen starkt eller svagt, fyrkortsstöd.
   \item[Sangbud] är \emph{naturliga}, på tvåtricksnivån starkt inviterande.
     \ebe

\paragraph{Exempel: (\spa{1})--\ru{2}--(\pass)} 

\bbe
\item[\hj{2}] Naturligt, rondkrav.
\item[\spa{2}] Klöver.
\item[\NT{2}] Naturligt, ej krav.
\item[\kl{3}] Ruterstöd, enkelhöjning eller starkt.
\item[\ru{3}] Ruterstöd, mittemellan.
\ebe

Högre bud är splinter i motståndarnas färg, anpassningsbud i ny färg.

\paragraph{Exempel: (\hj{2})--\kl{3}--(\pass)} 

\bbe
\item[\ru{3}] Naturligt, krav.
\item[\hj{3}] Spader.
\item[\spa{3}] Klöverstöd, bra hand.
\ebe

\paragraph{Exempel: (\kl{3})--\spa{3}--(pass)}

\bbe
\item[\NT{3}] Naturligt.
\item[\kl{4}] Ruter.
\item[\ru{4}] Hjärter.
\item[\hj{4}] Spaderstöd, slaminvit.
\item[\spa{4}] Normal höjning.
\ebe

\subsubsection{Tredje hand bjuder ny färg.}

Inga överföringar längre. \emph{Dubbelt} är negativt.

\section{Sangförsvar}

{\em Malmö} är det som gäller.

Vi balanserar {\em mycket frekvent} efter \NT{1}-\pass-\pass!

\section{Försvar mot svaga två\-öpp\-ning\-ar}

\emph{(Nytt, tagit bort 'BTC'!)}

Standard, Dubbelt är UD och \NT{2} ca 16--18 hp, balanserad, varefter
\kl{3} är högfärgsfråga.
 
%Modellen är {\em BTC}, dvs {\em Dubbelt} av \hj{2} lovar fyrkorts spader,
%medan \NT{2} är en {\em UD} utan fyrkorts spader. Efter spärren \spa{2}
%blir det tvärtom, fast frågan gäller hjärterinnehav eller ej.

Lebensohl används så att direkt färgbud är svagt, medan \NT{2} är
kommando till \kl{3} och lovar några poäng (minst ca 7 hp). Undantag:
\kl{3} lovar minst ca 7 hp.

\section{Försvar mot svaga tre\-öpp\-ning\-ar}

Normala \emph{UD}. \NT{3} är ``to play''.

\section{Försvar mot {\em Multi} \ru{2}}

{\em Dubbelt} lovar åtminstone en fyrkorts högfärg och en {\em
UD}-lämplig hand. Om tredje hand bjuder en högfärg så är fjärde hands
dubbling en UD, medan \pass\ är krav.

\section{Försvar mot stark klöver}

Huvudinriktningen  är destruktiv. Man bjuder dock även med bra
två\-färgs\-hän\-der. \emph{DONT} gäller.

\bbe
   \item[Dubbelt] visar \emph{klöver}.
   \item[\NT{1}] en lång färg.
\ebe

\section{Försvar mot dubbeltydig klöver.}

Ungefär som mot stark klöver. Med bra kort sätter man sig lämpligen i
busken. Färgbud på enläget dock konstruktiva.

