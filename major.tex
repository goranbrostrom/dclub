\chapter{H{\"o}gf{\"a}rgs{\"o}ppningarna}

I f\"orsta och andra hand lovar h{\"o}gf{\"a}rgs{\"o}ppningarna
minst femkortsfärg och obalanserade h{\"a}nder (ej \ford{5}{3}{3}{2})
och (8)9--14 hp.

Svaren p{\aa} \hj{1} och \spa{1} i f\"orsta och andra hand
bygger p{\aa} {\em {\"o}verf{\"o}ring}.

\section{Trumfst\"od}

Fyrkorts- och längre trumfstöd visas direkt på minst tretricksnivån eller
med \NT{2}, {\em
oavsett styrka}. Det är en elementär tillämpning av {\em lagen om totala
antalet stick} ({\sc Lagen}), som säger att det alltid är rätt att bjuda på
tretricksnivån med totalt nio trumf och på fyrtricksnivån med tio trumf.
Vi gör det svårt för motståndarna!

% Ett nytt begrepp, \under{3}, har införts. Det är helt analogt med
% \under{2} men utförs, som namnet antyder, på tretricksnivån. Det medför
% att två hoppande borttag har fått stryka på foten, \hj{3} efter
% spaderöppningen, och \ru{3} efter hjärteröppningen. Dessa \under{3}--svar
% lovar minst fyrkortsstöd och ca 8--10 hp, eller ca 8--9 förlorare.

Med exakt trekortsstöd håller vi oss huvudsakligen till tvåtricksnivån, också i
enlighet med {\sc Lagen}. Buden \spa{1} (på \hj{1}), \NT{1}, \spa{2} (på
\hj{1}) och 
\kl{3} (på \spa{1}) är de enda, som kan innehålla exakt trekorts trumfstöd.

\begin{beskriv}
 \item[Trekorts trumfstöd] visas (endast) med buden
\begin{beskriv}
   \item[\NT{1}] På \hj{1}: 8+ hp, 5+ spader, 0-3 hjärter. På \spa{1}: 5--7
     hp och trekortsstöd eller 8--14 hp, högst dubbelspader.
   \item[\under{2}] 8-12 eller 15+ hp, i senare fallet krav till
                       utgång.
   \item[\spa{2}] (på \hj{1}) invit, exakt trekortsstöd.
   \item[\kl{3}] (på \spa{1}) invit, exakt trekortsstöd.
\end{beskriv}
 \item[Fyrkorts trumfstöd] visas med buden
  \begin{beskriv}
   \item[\NT{2}] Minst 11 hp, minst fyrkortsst{\"o}d. Antalet förlorare
                bör inte överstiga åtta. Med högst sju förlorare
                tolereras mindre än 11 hp.
   \item[dubbelh{\"o}jning] 0--4 hp.
   \item[\under{3}] 5--8 hp.
   \item[Under \under{3}] 9--11 hp.
   \item[\NT{3}] (efter \hj{1}). Renons, minst fyrkortsstöd.
   \item[\la{4}] Renons, minst fyrkortsstöd.
 \end{beskriv}
 % \item[Femkorts trumfstöd] visas med buden
 %  \begin{beskriv}
 %   \item[\spa{3}] (efter \hj{1}). 10-13 hp, femkortsstöd, någon singelton.
 %   \item[\NT{3}] (efter \spa{1}). 10-13 hp, femkortsstöd, någon singelton.
 %   \item[hopp till utg{\aa}ng] 0-7 hp, minst femkortsstöd.
 % \end{beskriv}
\end{beskriv}

{\em Observera} att andra svarsbud förnekar trumfstöd, trekorts eller
längre.

\subsection{Stenbergs \NT{2}}

Svar p{\aa} \ho{1} -- \NT{2}:
\begin{beskriv}
   \item[\kl{3}] Slamvänlig, ej svitad tvåfärgshand.
      \bbe
         \item[--\ru{3}] Positiv invit
         \item[--andra hö] Positiv invit
         \item[--3 i öppningsfärgen] Minimum
         \item[--\NT{3}] Positiv invit, klöver.
       \ebe
   \item[\ru{3}] Positiv hand med positiv ruter. 
   \item[3 i \"oppningsf\"argen] Passbart.
   \item[3 i andra h\"ogf\"argen] Positiv hand med något i bjuden.
   \item[Hopp till utg{\aa}ng] Minst sexkorts trumff\"arg, d{\aa}ligt
                               med kontroller f\"or \"ovrigt. Mest spärr.
   \item[\NT{3}] Positiv med klöver.
   \item[\la{4}] Stark 5--5 typ.
\end{beskriv}

Allm\"ant g\"aller att \NT{3} {\em aldrig} \"ar f\"orslag till slutbud.
Efter positiva invitbud på tretricksnivån bjuder man singel på
fyrtricksnivån. Saknas singel går man ner i \NT{3}, varefter äkta och oäkta
kontroller bjuds huller om buller.

\section{{\"O}ppningsbudet \hj{1}}

Svar:
\begin{beskriv}
   \item[\NT{1}] 8+ hp, minst femkortsspader.
   \item[\spa{1}] (i) 8--14 hp, krav, 0--4 spader, 0--2 hjärter; (ii) 5--7
     hp, exakt trekorts hjärter. 
   \item[\kl{2}] Lovar ruterf\"arg eller stark balanserad hand. Förnekar
                 trekorts hjärter.
         \begin{nummer}
           \item 12--14 hp, minst sexkorts ruter. Bjuder \ru{3} på \ru{2}.
           \item 15+ hp, minst femkortsf{\"a}rg. Bjuder annat p{\aa} svaret
                      \ru{2}. Krav till utgång.
	   \item 16+ hp, balanserad (femkorts lågfärg är OK) eller
                        \ford{4}{1}{4}{4}. Krav till utgång.
          \end{nummer}
   \item[\ru{2}] Exakt trekorts hj\"arterst\"od.
          \begin{nummer}
            \item 8-12 hp, passar på \hj{2}.
            \item 16+ hp, krav till utgång.
           \end{nummer}
   \item[\hj{2}] Stark hand med kl\"over som
                 huvudf\"arg och och eventuellt fyrkorts sidofärg. Högst
                 dubbelhjärter.
   \item[\spa{2}] Invit med exakt trekortsstöd.
   \item[\NT{2}] Stenbergs, se ovan!
   \item[\kl{3}] 9--11 hp, minst fyrkortsstöd.
   \item[\ru{3}] 5--8 hp, minst fyrkortsstöd.
   \item[\hj{3}] 0--4 hp, exakt fyrkortsstöd.
   \item[\hj{4}] Spärr med svitad hand och utmärkt stöd. Kan av taktiska
     skäl någon gång bjudas med utgångshand utan slamhopp.
   %\item[\spa{3}] 10-13, femkorts hjärterstöd och någon singelton.
   %\item[\NT{3}] 10-13, minst fyrkortsstöd och spaderrenons.
   %\item[\la{4}] 10-13, minst fyrkortsstöd och renons.
\end{beskriv}

\subsection{\hj{1} - \spa{1}}

Svararen lovar högst 14 hp och 0--4 spader. Öppnarens återbud:

\begin{beskriv}
   \item[\NT{1}] Fyrkorts \emph{spader}.
   \item[\kl{2}] Minst fyrkorts klöver.
   \item[\ru{2}] Minst fyrkorts ruter.
   \item[\hj{2}] 9--13 hp, sexkorts hjärter.

   \item[\spa{2}] 13--14 hp, minst 6--4 i hjärter/spader..

   \item[\NT{2}] 13--14 hp, bra sexkortsfärg, ``balanserad'' typ.

   \item[\la{3}] 13--14 hp, bra 5--5.

   \item[\hj{3}] 13-14 hp, sexkortsf{\"a}rg.

\end{beskriv}

Notera att med femfemmor eller värre kan poängstyrkan vara över den
nominella 15-poängsgränsen; tvåfärgshänder kan komma bort efter en
\kl{1}-öppning.

\subsection{\hj{1} - \NT{1}}

Svararen lovar 8+ hp och 5+ spader. Naturlig fortsättning.

\subsection{\hj{1} - \kl{2}}

Lovar alltså ruter eller stark hand utan femkortsfärg. Förnekar trekorts
hjärterstöd.

Fortsatt budgivning:

\begin{beskriv}
   \item[\ru{2}] Minimum, dålig ruteranpassning eller brist på bättre.
   \begin{beskriv}
	\item[--\pass] Är inte tillåtet!!
	\item[--\hj{2}] Minst femkorts ruter, enfärgad, krav till
                        utgång.
	\item[- \spa{2}] Fyrkorts spader och femkorts ruter. Krav till utgång.
			\kl{3} av öppnaren är nu ''fjärde färg'', annars
			naturlig fortsättning.
	\item[- \NT{2}] Balanserad typ (kan innehålla dålig femkorts
                        lågfärg). Krav till utgång.
	\item[- \kl{3}] 4+ klöver och 5+ ruter, utgångskrav.
	\item[- \ru{3}] Passbart, 6+ färg.
	\item[- \hj{3}] Krav till utgång med bra tvåkortsstöd och 6 bra ruter.
	\item[- \spa{3}] 5--6 i hårda färgerna, utgångskrav.
   \end{beskriv}
   \item[\hj{2}] Minst sexkortsfärg, krav för en rond.
		\begin{beskriv}
		   \item[--\spa{2}, - \kl{3}] Fyrkortsfärg, 5+ ruter,
			krav till utgång.
		   \item[--\NT{2}] Kravinvit. Kan dölja starka kort.
		   \item[--\ru{3}] 6+ ruter, passbart. 
		   \item[- \hj{3}] H-x i hjärter.
		   \item[- \spa{3}, - \kl{4}] Singel, H-x i hjärter.
		\end{beskriv}
   \item[\spa{2}] Naturligt, krav till utgång.
   \item[\NT{2}] ``Balanserad'' 14--15 hp, krav till utgång.
   \item[\la{3}] Naturligt, krav till utgång.
   \item[\hj{3}] Krav med stark färg.
\end{beskriv}

\subsection{\hj{1} - \ru{2}}

Exakt trekortsstöd och 8..12 eller 16+ hp.
Fortsatt budgivning:
\begin{beskriv}
   \item[\hj{2}] Inga ambitioner mot den svaga höjningen, femkortsfärg. Med
                 trekortsstöd och ca 15-18 hp utan slamvisioner hoppar
svararen lämpligen direkt till utgång (\NT{3} är utgångsväljare). Med
trekortsstöd och slamvisioner måste man överväga trumfbyte, alltså inte
bjuda högre än \ru{3}. 
	\begin{beskriv}
	   \item[--\pass] Obligatoriskt med den svaga stödhanden.
	   \item[--\spa{2}] Äkta färg, krav till utgång.
	   \item[--\NT{2}] "Balanserad", krav till utgång.
                           Ofta slamintresse, öppnaren visar upp en
                           sidofärg för eventuellt trumfbyte.
	   \item[--\la{3}] Krav med femkortsfärg.
           %\item[--\hj{3}] trekorts hjärterstöd. 
	   %\item[- \spa{3}] Hoppande borttag, bestämmer trumfen.
	   \item[- \la{4}] Kortfärg.
	\end{beskriv}
   \item[\spa{2}, \la{3}] Positiv invit mot svaga stödhanden. Lovar
fyrkortsfärg med hänsyn tagen till att trumfbyte kan vara aktuellt.
Höjningar av den positiva inviten lovar alltså fyrkortsstöd och troligen
den starka stödhanden. 
   \item[\NT{2}] Stark hand utan sidofärg.
   \item[\hj{3}] Minst sexkorts hjärter, 9-13 hp. Svaga stödhanden måste
passa, övrigt naturligt, \NT{3} hjärterstöd
utan singel, slamintresse, \la{4} singel med hjärterstöd osv.
\end{beskriv}

\subsection{\hj{1} - \hj{2}}

Svararen kräver till utgång med klöver men utan hjärterstöd. Naturlig
fortsättning. 

\subsection{\hj{1} - \spa{2}}

Invit med trekortsstöd. Naturlig fortsättning.

\section{{\"O}ppningsbudet \spa{1}}

Svar:
\begin{beskriv}

   \item[\NT{1}] (i) 8--14 hp, h\"ogst dubbelspader; (ii) 5--7 hp, exakt
     trekortsstöd. 
   \item[\kl{2}] Lovar ruterf\"arg eller stark balanserad hand. Förnekar
     trekorts spader.
         \begin{nummer}
           \item 12--14 hp, minst sexkorts ruter.
                 Bjuder \ru{3} p{\aa} svaret \ru{2}.
           \item 15+, minst femkortsf{\"a}rg.
                      Bjuder vidare p{\aa} svaret \ru{2}.
	   \item 16+, balanserad eller \ford{1}{4}{4}{4}.
          \end{nummer}

   \item[\ru{2}] Lovar entydigt hj\"arterf\"arg, högst dubbelspader:
          \begin{nummer}
            \item 13--14 hp, minst sexkorts hj{\"a}rter.
            \item 15+, minst femkorts hj{\"a}rter.
           \end{nummer}

   \item[\hj{2}] Exakt trekortsstöd, 8--12 eller 16+ hp. 

   \item[\spa{2}] Krav till utgång med klöver som bästa färg men utan
     spaderstöd. 

   \item[\NT{2}] Stenbergs.
   \item[\kl{3}] Invit med exakt trekortsstöd.
   \item[h\"ogre bud] Analogt med svaren p{\aa} \"oppningsbudet \hj{1}.
\end{beskriv}

\subsection{\spa{1} - \NT{1}}

Fortsatt budgivning är naturlig.


\subsection{\spa{1} - \kl{2}}

Lovar ruterfärg eller stark hand utan (bra) femkortsfärg.
Fortsatt budgivning är tämligen analog med \hj{1} -- \kl{2}. (\spa{1} --
\kl{2}; \ru{2} -- \spa{2} är kravet med enfärgad ruter.) 

\subsection{\spa{1} - \ru{2}}

Lovar minst sexkorts hjärterfärg och 13+ hp eller femkortsfärg och 15+ poäng. 
Inga märk\-vär\-dig\-he\-ter i fortsättningen,
eftersom budet är tämligen väl\-de\-fi\-ni\-e\-rat:

\bbe
   \item[\hj{2}] Neutralt, ej passbart. Högst trekorts hjärter. 
	\bbe
	   \item[--\spa{2}] Bra dubbelton, krav.
	   \item[--\NT{2}] Balanserad utan spaderstöd, krav.
	\ebe
   \item[\spa{2}] Minst sexkortsfärg, bra färg eller eller hjärteranpassning.
		 Ej passbart: 
		\bbe
		   \item[--\NT{2}] Kravinvit.
		   \item[--\la{3}] Naturligt krav.
		   \item[--\hj{3}] Minimum med 6+ hjärter.
		   \item[--\spa{3}] Invit.
		\ebe
   \item[\NT{2}] ``Balanserad'' 14--15 hp, 2-3 hjärter, {\em eller}
fyrkortsstöd. Krav till utgång. 
   \item[\la{3}] Naturligt.
   \item[\hj{3}] Fyrkortsstöd, krav. Lovar dock ej särskild styrka.
   \item[\spa{3}] Naturligt, krav.
   \item[\la{4}] Fyrkortsstöd, kortfärg, något tillägg. 
\ebe

\subsection{\spa{1} - \hj{2}}

Fortsatt budgivning är helt analog med \hj{1} -- \ru{2}.

\subsection{\spa{1} - \spa{2}}

Fortsättningen helt analog (och naturlig) med densamma efter \hj{1} -- \hj{2}:
