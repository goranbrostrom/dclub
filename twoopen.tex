\chapter{Tv\aa{}\"oppningarna}

\section{\"Oppningsbudet 2 kl\"over}

Detta {\"o}ppningsbud undviks i g{\"o}rligaste m{\aa}n genom att 
man {\"o}ppnar med
\NT{1} eller \kl{1} d{\aa} man har \ford{5}{4}{2}{2}, 
fem\-korts\-kl{\"o}ver och {\aa}tminstone n{\aa}gon
gubbe i tv{\aa}\-korts\-f{\"a}r\-ger\-na.

\subsection{Svar p\aa{} \kl{2}}
\bbe
  \item[-\ru{2}] Visar minst 8 hp, konventionellt fr{\aa}gande.

   \item[-\ho{2}] Invit med femkortsf{\"a}rg, ej krav, men {\"o}ppnaren
                   tar varje chans att bjuda en g{\aa}ng till\footnote{\em 
                   Pass b{\"o}r visa minimum, dubbelton i st{\"o}d och 
                   d{\aa}lig kl{\"o}verf{\"a}rg och f{\"o}rmodligen 
                   f{\"o}rneka fyrkorts spader}.

   \item[-\NT{2}] Rondkrav med
     \bnu
      \item exakt femkorts h{\"o}gf{\"a}rg ({\"a}ven 5-4, 4-5 och 5-5 i
            h{\"o}gf{\"a}rgerna
            och utg{\aa}ngskrav, eller
      \item utg{\aa}ngsinvit med klöverstöd.
     \enu

   \item[-\kl{3},-\kl{4}] Sp{\"a}rr. \kl{4} visar en Yarborough med tre- eller
                    fyrkortsst{\"o}d.
   \item[-\ru{3}, -\ho{3}] Hoppande borttag med {\em mycket} bra minst
                      sexkorts\-f{\"a}rg. Naturlig fort\-s{\"a}ttning, 
                      \ru{4} av
                      {\"o}ppnaren {\"a}r accept av svararens f{\"a}rg med
                      slam\-v{\"a}n\-liga kort.
   \item[-\ru{4}] Kl{\"o}veraccept med ruterkontroll
\footnote{Utan ruterkontroll bjuder man \ru{2} f{\"o}ljt av \kl{4}.}.
\ebe

\subsection{\kl{2} - \ru{2}}
   \bbe

     \item[\hj{2}] Lovar n{\aa}gon fyrkorts h{\"o}gf{\"a}rg.
       \bbe
         \item[-\spa{2}] Fr{\aa}gar efter vilken,
           \bbe
             \item[\NT{2}] hj{\"a}rter, 
             \item[\kl{3}] spader, 
            \ebe
                       varefter \ru{3} {\"a}r trumfst{\"o}d och 
                       resten naturligt.
         \item[-\NT{2}] {\"a}r invit med 10-11 hp.
         \item[-\kl{3}] lovar en bra enkel h\"ojning.
         \item[-\ru{3}] minst femkortsf\"arg, krav till utg\aa{}ng.
         \item[-\ho{3}] minst sexkortsf{\"a}rg, krav till utg\aa{}ng.
       \ebe

     \item[\spa{2}] visar minimum, ingen fyrkorts h{\"o}gf{\"a}rg 
                  (sexkorts kl{\"o}ver).
        \bbe
           \item[-\NT{2}, -\kl{3}] \"ar avl{\"a}gg.
           \item[-\ru{3}] fr{\aa}gar efter h{\"o}gf{\"a}rgsh{\aa}ll, 
           \item[-\ho{3}] lovar sexkortsf{\"a}rg och \"ar krav.
        \ebe

     \item[\NT{2}] Maximum, ingen fyrkorts h{\"o}gf{\"a}rg, 
                 tv{\aa} sidoh{\aa}ll.
         \bbe
             \item[-\ru{3}] fr{\aa}gar efter h{\"o}gf{\"a}rgsh{\aa}ll 
                            (\NT{3} h{\aa}ll i b{\aa}da).
             \item[-\ho{3}] \"ar krav med sexkortsf{\"a}rg.
         \ebe

     \item[\kl{3}] Maximum, ingen fyrkorts h{\"o}gf{\"a}rg, ett sidoh{\aa}ll.
          \bbe
             \item[-\ru{3}] fr{\aa}gar som vanligt efter 
                          h{\"o}gf{\"a}rgsh{\aa}ll.
             \item[-\ho{3}] {\"a}r krav med sexkortsf{\"a}rg.
          \ebe

     \item[\ru{3}] Maximum, minst fyrkortsfärg.
     \item[\ho{3}] Femkortsf{\"a}rg och f{\"o}ljaktligen (minst) sexkorts
                     kl{\"o}ver.

     \item[\NT{3}] Maximum, g{\aa}ende (minst) sexkorts kl{\"o}ver.
   \ebe

\subsection{\kl{2} - \NT{2}}
 Lovar allts\aa\ en invithand med klöverstöd
 {\em eller} en kravhand med femkorts
 h\"og\-f\"arg.
 {\"O}ppnaren f{\"o}rut\-s{\"a}tter den balan\-serade vari\-anten och 
  bjuder \kl{3} som av\-slag, {\"o}vrigt \"ar accept: 
\bbe
   \item[\kl{3}] {\"a}r avslag av inviten.
   \item[\ru{3}] accept, ingen fyrkorts h{\"o}gf{\"a}rg, 
   \item[\hj{3}] accept, fyrkortsf{\"a}rg,
   \item[\spa{3}] accept, fyrkortsf{\"a}rg, f{\"o}rnekar trekortshj{\"a}rter, 
   \item[\NT{3}] 4-3 i h{\"o}gf{\"a}rgerna (dvs \ford{4}{3}{1(0)}{5(6)}, 
               uppifr{\aa}n).
\ebe

\section{\"Oppningsbudet \ru{2}}

9--14 hp, 4--4 (ev.\ 3--4) i högfärgerna, 4--5 kort i klöver. Svararen
bjuder

\bbe
\item[--\pass, \ho{2},\kl{3}] Slutbudsförslag. Maximal hand med femkorts
  klöver får bjuda vidare. 
\item[--\NT{2}] Fråga efter fördelning och styrka:
\bbe
    \item[\kl{3}] minimum,
\bbe
\item[--\ru{3}] utgångskrävande relä; med \ford{3}{4}{1}{5} /
  \ford{4}{4}{0}{5} hos öppnaren 
fort\-sät\-ter budgivningen som nedan; med \ford{4}{4}{1}{4} bjuder öppnaren
\NT{3}. Här 
är \ru{4} och \hj{4} överföring till högfärg som inte kan fastställas med
kravtempo. 
 \item[--annat] Naturligt krav till utgång.
\ebe
    \item[\ru{3}] tillägg, \ford{4}{4}{1}{4},
    \item[\hj{3}] tillägg, \ford{3}{4}{1}{5},
    \item[\spa{3}] tillägg, \ford{4}{4}{0}{5},
\ebe
Svararen kan passa på \kl{3}, men när öppnaren visar tillägg råder utgångskrav.
Svararens \ho{3}, \kl{4} fastställer trumfen. När högfärg inte kan
fastställas med 
kravtempo (dvs efter \ho{3}) används \ru{4}/\hj{4} som överföring. Höjning
till \ho{4} 
är aldrig överföring; färgen över kan ju bjudas naturligt.

\item[--\ru{3}] Naturligt utgångskrav.
\item[--\ho{3}, \kl{4}] Naturligt inviterande.
\ebe

I störd budgivning
är Dubbelt straffdubbling, färgbud okrav, \NT{2} som ostört.

\section{\"Oppningsbuden 2 hjärter och 2 spader}

Svaga två, svaren på \NT{2} enligt \emph{Ogust}.

\section{\"Oppningsbudet 2 sang}

Stark lågfärgsspärr

\section{Spärr på treläget}

Är tämligen modernt.

\section{\"Oppningsbudet 3 sang}

En \emph{stark} högfärgsspärr. Behöver utvecklas.

