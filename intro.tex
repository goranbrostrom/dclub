\chapter{Inledning}

\section{N\aa{}gra allm\"anna principer}

\subsection{Kl\"overbudgivningens ''långa v\"ag''}

G{\aa}r ut p{\aa} att b{\aa}da
     bjuder n\"armsta f\"argbud utan n{\aa}got s\"arskilt att ber\"atta.

     F\"or {\em \"oppnaren}\/s del inneb\"ar det att han avviker med {\em 
     minimal styrka} eller {\em ovanlig f\"ordelning}.

     F\"or {\em svararen}\/s del inneb\"ar det att han avviker med 
    {\em maximal styrka} eller ovanlig fördelning.

\begin{tabbing}
xxxx\= xxxxxxxxxx\=xxxxxxxxxxxxxxxxxxxxxxxxxxx\=xxxxxxxxxxxxxxxxxxxxxxx \kill
\>\kl{1} \>        \> 9--11 eller 15+ hp. \\
\\
\>       \> --\ru{1}  \> 0--8 hp. \\
\\
\>\hj{1} \>        \> Minst 18 hp om balanserad. \\
\>       \>        \> H\"ogst 14 mp om obalanserad. \\
\\
\>       \> --\spa{1} \> 0--6 hp. \\
\\
\>\kl{2} \>        \> Inviterande med klöver och obalanserad \\
\>       \>        \> eller 25+ hp, balanserad. \\
\\
\>       \> --\ru{2} \> 0--3 hp i AKQ och ingen egen bra färg. \\
\\
\>\hj{2} \>        \> Minst 23 hp om balanserad, \\
\>       \>        \> klöver och hjärter om obalanserad. \\
\\
\>       \> --\spa{2} \> Vilket?\\
\\
\> \NT{2} \>   \> 25+ hp, balanserad. Krav till utgång.
\end{tabbing}

\subsection{N\"ar tr\"ader vanliga sangsystemen i kraft efter \kl{1}?}

Jo, n\"ar \"oppnaren \aa{}terbjuder \NT{1}\ eller \NT{2} naturligt
utan att svararen har visat n\aa{}got om sin f\"ordelning.

\subsection{Konventionella kontra naturliga bud}

 Ett \emph{icke definierat} bud tolkas
      i f\"orsta hand {\em na\-tur\-ligt} och {\em begr\"ansat} (ej krav),
      snarare \"an konventionellt och starkt (krav).

\subsection{Fj\"arde f\"arg}

P{\aa} ett fj\"ardef\"argkrav bjuder \"opnaren s{\aa} naturligt som m\"ojligt.
Billigaste ombud av visad f\"arg visar minimum utan h{\aa}ll i fj\"arde
f\"argen, medan 
bud i fj\"arde f\"argen lovar till\"agg utan h{\aa}ll, inga extra
f\"arg\-l\"ang\-der 
(t.\ ex.\ \ford{2}{5}{4}{2}). \NT{2} \"ar krav f\"or en rond, men visar
inte n\"odv\"andigtvis till\"agg ut\"over h{\aa}llet.

Notera att i vissa sekvenser efter öppningen \ru{1} (till exempel \ru{1} --
\spa{1}; \NT{1}) så är budet \spa{2} konventionellt krav och lovar ingen
extra längd i spader.

\subsection{Absoluta dubblingar}

I \emph{alla} lägen som inte är annorlunda definierade så visar
\emph{Dubbelt} korthet i färgen. Man blir inte ledsen om partnern står
med, så idealt har man dubbelton i färgen man dubblar (helst inte renons). 

%\subsection{Mancoff}

%N\"ar en av motst{\aa}ndarna dubblar en h{\aa}llfr{\aa}ga eller ett
%fj\"ardef\"argbud, s{\aa} anv\"ands f\"oljande svar:

%\begin{description}

%   \item[Pass] halv-- eller helh{\aa}ll, varefter {\em RD} fr{\aa}gar efter
%                helt h{\aa}ll.
%   \item[RD] inget h{\aa}ll, men minst tv{\aa} kort i f\"argen.
%   \item[Sang] dubbelh{\aa}ll (eller enkelh{\aa}ll, om man \"ar beredd att
%                  spela sang \"and{\aa}).
%   \item[Övrigt] singel eller renons i den dubblade f\"argen. Om man redan
%                 {\em lovat} en balanserad hand, s{\aa} visar man nu
%                 Ess--hacka). 
%\end{description}

\subsection{Dubbling av kontrollbud}

D{\aa} en motst{\aa}ndare dubblar ett kontrollbud visar {\em RD} egen
f\"orstakontroll, {\em bud} hj\"alp i f\"argen (t. ex. {\em Damen}), och
\pass\ 
ingen hj\"alp. Om d{\aa} kontrollbjudaren redubblar, s{\aa} visar det att
situationen \"ar under kontroll, t.\ ex.\ med singel eller {\em KD}.

\subsection{Fyrkorts trumfst\"od}

 i h\"ogf\"arg skall visas {\em omg\aa{}ende}.

\subsection{Nyheter i denna utg{\aa}va av systemet}

Utg{\aa}vans aktualitet best\"ams av versionsnummer och datum p{\aa}
titelbladet. Nyheter noteras i förordet och inte med
markeringar i löpande text.

\newpage
\section{\"Oppningsbud, en \"oversikt}

\subsection{Första- och andrahandsöppningar}

\bbe
   \item[\kl{1}] Svagt, 9--11 hp, eller starkt, minst 15 hp: 
     \bnu
       \item 9--11 hp, ``balanserad''.  Vissa semibalanserade typer tillåts,
t.\ ex.\ \marmic\ med singelruter, \ford{4}{2}{2}{5} och \ford{2}{3}{2}{6}.
Femkorts högfärg är tillåtet. 
       \item Balanserad hand med 15+ hp.
       \item Obalanserad hand med minst 15 \emph{bra} hp.
       \item Obalanserad hand med högst 14 mp.
     \enu

   \item[\ru{1}] 9--14 hp, obalanserad hand, minst fyrkorts ruter. Kan
               inneh{\aa}lla l\"angre kl\"over med minimum. Om \marmic\ så
               minst 10 hp.
   \item[\hj{1}] 8--14 hp, minst femkortsf\"arg.
   \item[\spa{1}] 8--14 hp, minst femkortsf\"arg.
   \item[\NT{1}] 12--14 hp, balanserad hand. Femkorts högfärg tillåtet.

   \item[\kl{2}] 9--14 hp och
    \bnu
       \item minst sexkorts kl\"over.
       \item minst femkorts kl\"over och n{\aa}gon fyrkorts h\"ogf\"arg.
       \item 13--14 hp och minst femkorts kl\"over och minst fyrkorts ruter.
    \enu

   \item[\ru{2}] Dubbeltydigt:
     \bnu
       \item 5--7 hp, 5--6 hjärter.
       \item 5--7 hp, 5--6 spader.
     \enu

   \item[\hj{2}] 9--14 hp med exakt 4--4 i högfärgerna och 0--1 ruter.

   \item[\spa{2}] 5--7 hp, 5--5 med klöver.
   \item[\NT{2}] 5--7 hp, 5--5 utan klöver.
   \item[\la{3}\ho{3}] Naturlig spärr.
   \item[\NT{3}] Sp\"arr med l{\aa}ng (ej genomg{\aa}ende) l{\aa}gf\"arg.
   \item[\la{4}] Sydafrikansk Texas.
\ebe

\subsection{Tredje- och fj{\"a}rdehands{\"o}ppningar}

Tredje- och fjärdehandsöppningar är anpassade till att man vet att partnern
har högst åtta/nio hp; i princip visar alla öppningsbud tre poäng mer an i
första och andra hand, och svarshansden drar av tre poäng. Vissa
justeringar måste dock av naturliga skäl göras. Tilläggas bör också att vi
tillåter vissa friheter i tredje hand.


% Entydigt stark klöver
% (15+ hp), övriga öppningar endast syftande till att snabbt hitta delis och, i
% tredje hand, att störa. Enöppningar naturliga, oftast enfärgshand,
% \NT{1} enfärgad klöver. \kl{2} = kl + hö, \ru{2} = ru + hj, \hj{2} = hj +
% sp, \spa{2} = sp + ru, \NT{2} = 21-22 hp, balanserad.
 
% \bbe
%    \item[\kl{1}] 15+ hp, dock ej 21-22 hp balanserad.
% 	Fortsatt budgivning ungefär som efter \kl{1} i första och andra hand.

%    \item[\ru{1}\ho{1}] ``Längsta färg.'', 10-15 hp.

%    \item[\NT{1}] 10-15 hp, enfärgad klöver eller klöver och ruter.

%    \item[\kl{2}] 10-15 hp, minst femkorts klöver och fyrkorts högfärg.

%    \item[\ru{2}] 10-15 hp, minst femkorts ruter och fyrkorts hjärter.
 
%    \item[\hj{2}] 10-15 hp, minst femkorts hjärter och fyrkorts spader.

%    \item[\spa{2}] 10-15 hp, fyrkorts spader och minst femkorts ruter.  

%    \item[\NT{2}] Balanserad 21-22 hp.
% \ebe
