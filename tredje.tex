\chapter{I tredje och fjärde hand}

\section{Inledning}

När man överväger en tredje- eller fjärdehandsöppning måste man ta
hänsyn till att partnern {\em inte} öppnade. Han kan alltså inte ha
mer än 7--8 hp. Han har inte heller
en svag tvåöppning i högfärg. vår slutsats är att 
 klöveröppningen bör vara \emph{entydigt stark}, (17)18+ hp.

Övriga öppningar är alltså begränsade uppåt till ca 17 hp, och det främsta
mäålet för dem är kamp om delkontraktet. Den situationen är olika i tredje
och fjärde hand, därför behandlas de var för sig.

\section{Öppningen \kl{1}}

Klöveröppningen är samma i tredje och fjärde hand men skiljer sig från den
i första och andra hand. Den visar alltså 18+ hp, alla fördelningar. Svaren är
dock så långt möjligt samma som efter öppning i första och andra
hand. Efter annat svar än \ru{1} råder krav till utgång. 

\bbe

\item[--\ru{1}]
\bnu
  \item 0--6 hp, alla fördelningar. Sju bra poäng innehåller två
  kungar, ett ess och en dam, en kung och dam-knekt i samma färg, etc, dvs
  inte tre spridda knektar och två damer. Typ.

\item 7--8 hp, 6+ \emph{ruter}, krav till utgång.
\enu
  
\item[--\hj{1}] 7--8 hp, ``resten''. Krav till utgång, naturlig fortsättning.

\item[--\spa{1}] 7--8 hp, balanserad, förnekar femkorts högfärg. Ersätter
  budet \NT{1}. Krav till utgång, naturlig fortsättning.

\item[--\NT{1}] 5--6 hp, 5--5 i lågfärgerna. Naturlig fortsättning, som kan
  stanna under utgång.

\item[--\kl{2}] 5--6 hp, 6+ bra \emph{klöver}. Stark utgångsinvit.
\item[--\ru{2}] 5--6 hp, 6+ bra \emph{hjärter}. Stark utgångsinvit, krav
  till \hj{3}.
\item[--\hj{2}] 5--6 hp, 6+ bra \emph{spader}. Stark utgångsinvit, krav
  till \spa{3}.
\item[--\spa{2}] 7--8 hp, 5--5 i lågfärgerna, krav till utgång.  
    
\ebe

\section{Tredjehandsbudgivning}

Här behandlas andra öppningsbud än \kl{1}.

I tredje hand öppnar vi fritt med fyrkortsfärger,
  kombinerat med 
tvåfärgsvisningar (minst 5--5, styrkan är ``varierande'', dvs låg) på tvåläget:

\bbe
\item[\ru{1}] Kan vara på tre kort om under 15 hp och bal med
  klöver. Återbudet \NT{1} visar \emph{fortfarande} klöver enligt
  systemet. Med balanserad hand passar man ev. på svaret \ho{1}. 
\item[\NT{1}] 15-17 hp, som vanligt.
\item[\kl{2}] Klöver plus en högfärg, 8--12 hp.
\item[\ru{2}] Högfärgerna, 8--12 hp.
\item[\hj{2}] Hjärter och ruter, 8--12 hp.
\item[\spa{2}] Spader och ruter, 8--12 hp .
\item[\NT{2}] Lågfärgerna 8--14 hp (kan vara något starkare än andra
  tvåfärgsvisande bud).
\ebe

Fortsatt budgivning är naivt naturlig (men enligt standard), men 2-över-1 lovar minst tvåstöd
förutom färgen man bjöd, hopp är minisplinter med fyrstöd och max.

\hj{1}--\NT{1} = 5+ spader.

Högre öppningsbud är spärr som vanligt.  

\section{Fjärdehandsbudgivning}

Här behandlas andra öppningsbud än \kl{1}.

I fjärde hand är vi striktare än i tredje hand. Enöppningar visar ca 13+
hp, femkortsfärg, sangöppningen visar 15--17 hp, balanserad. Fortsättningar
som i första och andra hand, även \hj{1}--\NT{1} = 5+ spader. Dock:
minisplinter och 2-över-en, som i tredje hand. 

Tvåöppningarna är som i tredje hand, men mer konstruktiva: Man ska ha hopp
om hemgång i de kontrakt man kan hamna i.

Vi får samla mer erfarenhet innan vi slutgiltigt definierar
uppföljningar. Tills vidare gäller ``naivt naturligt''.
