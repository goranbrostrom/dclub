\chapter{Störda en{\"o}ppningar}

Gäller öppningsbuden \ru{1}, \hj{1} och \spa{1}. Efter \NT{1}- och
\kl{1}-öppningarna, se respektive kapitel.
 
Efter intervention av motståndarna lovar \NT{2}, med eller utan hopp,
trumf\-stöd och minst invit till utgång.

\section{Efter mellankommande dubbling}

Överföring hela vägen! RD visar färgen över eller ingen ann lämplig
fortsättning, ca 10+ hp. Fortsatta dubblingar är UD.

\subsection{\ru{1} - (D)}

\bbe
\item[RD] Ca 10+ hp, ingen annan lämplig fortsättning.
\item[\hj{1}] 4+ spader, ca 8+ hp.
\item[\spa{1}] ca 8--10 hp, balanserad eller 5+ klöver.  
\item[\NT{1}] Bra enkelhöjning i ruter, 4-kortsstöd. 
\item[\kl{2}] Svag enkel höjning i ruter, ofta trekorts.
\item[\ru{2}] Hjärterspärr.
\item[\hj{2}] Spaderspärr.
\item[\spa{2}] Sanginvit.
\item[\NT{2}] Stark med ruterstöd, minst invit. 
\item[\kl{3}] Ruterspärr, 5+ ruterstöd.
\item[\ru{3}] Hjärterspärr.
\item[\hj{3}] Spaderspärr.
\item[etc, etc]
\ebe

\subsection{\hj{1} - (D)}
\bbe
\item[RD] Ca 10+ hp, ingen annan lämplig fortsättning.
\item[\spa{1}] ca 8--10 hp, balanserad eller 5+ klöver.  
\item[\NT{1}] Bra enkelhöjning med trekortsstöd.
\item[\kl{2}] 5+ ruter.
\item[\ru{2}] Svag enkelhöjning med trestöd.
\item[\hj{2}] Spaderspärr.
\item[\spa{2}] Sanginvit (alt. klöver?).
\item[\NT{2}] Stenbergs.
\ebe 

\subsection{\spa{1} - (D)}
\bbe
\item[RD] Ca 10+ hp, ingen annan lämplig fortsättning.
\item[\NT{1}] Bra enkelhöjning med trekortsstöd.
\item[\kl{2}] 5+ ruter.
\item[\ru{2}] 5+ hjärter.
\item[\hj{2}] Svag enkelhöjning med trestöd.
\item[\spa{2}] Sanginvit (alt klöver?)
\item[\NT{2}] Stenbergs.
\ebe 


\section{Inkliv på entricksnivån}

Fortfarande överföringsprinciper enligt ovan; D är ``catch-all-budet'', ca
10 hp. Fortsatta dubblingar är UD.

\section{Högre inkliv}

\begin{itemize}

\item \NT{2} är trumfstöd, även efter en ruteröppning.

%\item Resten är enligt modern standard, med negativa dubblingar.

\end{itemize}

% \section{Naturliga sanginkliv}

% \begin{itemize}

% \item {\em Dubbelt} är straffdubbling, ca 12+.

% \item \kl{2} är högfärgsfråga. Efter en högfärgsöppning kan man till
% exempel ha trekortsstöd och fyra-fem kort i andra högfärgen.

% \item Övriga färgbud är naturliga.

% \item \NT{2} Minst invit med fyrkorts trumf\-stöd.

% \end{itemize}

% \section{Inkliv av fjärde hand efter 1-över-1}

% \begin{itemize}

% \item \pass\ visar minimum och normalt högst tvåkortsstöd.

% \item {\em Dubbelt} visar trestöd.

% \item Ny färg är naturligt.

% \item Hoppstöd är minimum med fyrkortsstöd.

% \item \NT{2} är fyrkorts trumfstöd med tillägg. 

% \item \NT{1}  är sang, dvs lovar håll.

% \end{itemize}

% \section{Fjärde hand dubblar efter 1-över-1}

% Trestöd visas med en redubbling. För övrigt tämligen naturligt.

% \section{Fjärde hand kliver in efter \NT{1}}

% Dubbelt är straffdubbling. Dubbelt av sangbjudaren är mer åt UD-hållet, har
% dock minst dubbelton i inklivsfärgen.

% \section{Fjärde hand dubblar efter \NT{1}}

% Som ostört, dock

% \begin{itemize}

% \item öppnaren passar med minimum utan bra fördelning.
% \item {\em RD} visar en stark, (semi)balanserad hand med häktningsintresse.
% \end{itemize}
