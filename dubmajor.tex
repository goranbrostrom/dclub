\chapter{Störda en{\"o}ppningar}

Gäller öppningsbuden \ru{1}, \hj{1} och \spa{1}. Efter \NT{1}- och
\kl{1}-öppningarna, se respektive kapitel.
 
Efter intervention av motståndarna lovar \NT{2}, med eller utan hopp,
trumf\-stöd och minst invit till utgång.

I övrigt är jag inte säker på hur det bör vara. Jag föreslår följande enkla
utväg tills vidare:

\section{Efter mellankommande dubbling}

\begin{itemize}
\item Sangbud utan stöd undviks (redubbla eller bjud \spa{1} i stället). 
\item ``The
Kaplan interchange'' utgår. 
\item Överföring på entricksnivån, 
\item 1NT lovar stöd, trekorts efter \ho{1}, tre- eller fyrkorts efter
  \ru{1}. Svagt, ca 5--8. \under{2} är något starkare, ca 8--12 (plus en
  annan betydelse).  
\item Bud på treticksnivån och högre är ``minisplinter''. 
\end{itemize}

\subsection{\ru{1}--(D)}
\bbe
\item[RD] 4+ hjärter.
\item[\hj{1}] 4+ spader.
\item[\spa{1}] Sang.
\item[\NT{1}] 3-4 ruter, 5--8.
\item[Resten] Som ostört.
\ebe
Accept av överföringen visar trekortsstöd, passbart. Annat är som ostört.
Om motståndarna inte hör av sig mer så fortsätter vi som ostört.
\subsection{\hj{1}--(D)}
\bbe
\item[RD] 4+ spader.
\item[\spa{1}] Sang.
\item[\NT{1}] Trestöd, 5--8 hp (eller invit?).
\item[Resten] som ostört.
\ebe

\subsection{\spa{1}--(D)}
\bbe
\item[RD] Sang
\item[\NT{1}] Trestöd, 5--8 hp (eller invit?).
\ebe
%
%Överföring \emph{upp till enkel höjning}. RD visar färgen över eller ingen
% annan lämplig 
% fortsättning, ca 10+ hp. Fortsatta dubblingar är UD.

% \subsection{\ru{1} - (D)}

% \bbe
% \item[RD] Ca 10+ hp, ingen annan lämplig fortsättning.
% \item[\hj{1}] 4+ spader, ca 8+ hp.
% \item[\spa{1}] 5+ klöver, minst ca 7 hp.  
% \item[\NT{1}] Bra enkelhöjning i ruter, 4-kortsstöd. 
% \item[\kl{2}] Svag enkel höjning i ruter, ofta trekorts.
% \item[\ru{2}] Multi, som ostört.
% \item[\hj{2}] Som ostört.
% \item[\spa{2}] Som ostört.
% \item[\NT{2}] Stark med ruterstöd, minst invit. 
% \item[\kl{3}] Minisplinter.
% \item[\ru{3}] Som ostört.
% \item[\ho{3}] Minisplinter.
% \item[etc, etc]
% \ebe

% \subsection{\hj{1} - (D)}
% \bbe
% \item[RD] Ca 10+ hp, ingen annan lämplig fortsättning (högst
%   dubbelhjärter). Kan innehålla spaderfärg.
% \item[\spa{1}] 5+ klöver, minst ca 7 hp.  
% \item[\NT{1}] Inviterande enkelhöjning med trekortsstöd.
% \item[\kl{2}] 5+ ruter, ca 7+ hp.
% \item[\ru{2}] Svag (8--11) eller stark enkelhöjning med trestöd (som ostört).
% \item[\hj{2}] Dålig stödhand (5--7).
% \item[\spa{2}] Minisplinter.
% \item[\NT{2}] Inviterande Stenbergs.
% \item[\la{3}] Minisplinter.
% \ebe 

% \subsection{\spa{1}--(D)}
% \bbe
% \item[RD] Ca 10+ hp, ingen annan lämplig fortsättning (högst dubbelspader).
% \item[\NT{1}] Inviterande enkelhöjning med trekortsstöd.
% \item[\kl{2}] 5+ ruter, ca 7+ hp.
% \item[\ru{2}] 5+ hjärter, ca 7+ hp.
% \item[\hj{2}] Svag (8--11) eller stark enkelhöjning med trestöd (som ostört).
% \item[\spa{2}] Dålig stödhand (5--7).
% \item[\NT{2}] Stenbergs.
% \item[\hj{3}, \la{3}] Minisplinter.
% \ebe 


\section{Inkliv på entricksnivån}

%Fortfarande överföringsprinciper enligt ovan \emph{upp till inklivsfärgen};
%D är ``catch-all-budet'', ca 10 hp. Fortsatta dubblingar är UD.

\subsection{\ru{1}--(\hj{1})}

\bbe
\item[D] Spader
\item[\spa{1}] En typ av balanserad (bjud \NT{1}).
\item[\NT{1}] höjning i ruter.
%\item[\kl{2}] Enkelhöjning i ruter.
%\item[\ru{2}] ``Multi'', spader eller ruterhöjning.
%\item[\hj{2}] Klöver.
%\item[\spa{2}] Fyrkorts spader.
\ebe
Resten som ostört.

\subsection{\ru{1}--(\spa{1})}

\bbe
\item[Dubbelt] ``bjud \NT{1} (med en normal hand)''. 
\item[\NT{1}] Ruterstöd.
\ebe
Resten som ostört.

\subsection{\hj{1}--(\spa{1})}

\bbe
\item[D] UD, högst dubbelhj.
\item[\NT{1}] Trestöd.
\item[\kl{2}] Ruter, som ostört (med stark balanserad, Dubbla).
\ebe

Resten som ostört.

\section{Högre inkliv}

Inga överföringar (än). Efter inkliv med \NT{1} är \kl{2} högfärgsfråga.

\begin{itemize}

\item \NT{2} är trumfstöd, även efter en ruteröppning.

%\item Resten är enligt modern standard, med negativa dubblingar.

\end{itemize}

% \section{Naturliga sanginkliv}

% \begin{itemize}

% \item {\em Dubbelt} är straffdubbling, ca 12+.

% \item \kl{2} är högfärgsfråga. Efter en högfärgsöppning kan man till
% exempel ha trekortsstöd och fyra-fem kort i andra högfärgen.

% \item Övriga färgbud är naturliga.

% \item \NT{2} Minst invit med fyrkorts trumf\-stöd.

% \end{itemize}

% \section{Inkliv av fjärde hand efter 1-över-1}

% \begin{itemize}

% \item \pass\ visar minimum och normalt högst tvåkortsstöd.

% \item {\em Dubbelt} visar trestöd.

% \item Ny färg är naturligt.

% \item Hoppstöd är minimum med fyrkortsstöd.

% \item \NT{2} är fyrkorts trumfstöd med tillägg. 

% \item \NT{1}  är sang, dvs lovar håll.

% \end{itemize}

% \section{Fjärde hand dubblar efter 1-över-1}

% Trestöd visas med en redubbling. För övrigt tämligen naturligt.

% \section{Fjärde hand kliver in efter \NT{1}}

% Dubbelt är straffdubbling. Dubbelt av sangbjudaren är mer åt UD-hållet, har
% dock minst dubbelton i inklivsfärgen.

% \section{Fjärde hand dubblar efter \NT{1}}

% Som ostört, dock

% \begin{itemize}

% \item öppnaren passar med minimum utan bra fördelning.
% \item {\em RD} visar en stark, (semi)balanserad hand med häktningsintresse.
% \end{itemize}
